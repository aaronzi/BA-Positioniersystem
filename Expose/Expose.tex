\documentclass[12pt, a4paper, twoside]{article} %Dokumentenklasse Setzen

\usepackage[a4paper, left=2.5cm, right=2.5cm]{geometry} %Seitenrand setzen

\usepackage[T1]{fontenc}
\usepackage{lmodern}
\usepackage[utf8]{inputenc}
\usepackage[ngerman]{babel}
\usepackage{csquotes}
\usepackage{xpatch}
\usepackage{xspace} % setzten von Leerzeichen nach Abkürzungen
\usepackage{microtype} % für glättere Seitenränder

%Mathe Pakete
\usepackage{amsmath}
\usepackage{thmtools}
\usepackage{amsfonts}
\usepackage{amssymb}

%Listenumgebungen
\usepackage{listings}
\usepackage{paralist}
\usepackage{enumitem}

%Demo Text
\usepackage{blindtext}

% Farb-Pakete
\usepackage{xcolor}
\usepackage{colortbl}

% Für erweiterte Tabellen
\usepackage{longtable}

% Für Einheiten
\usepackage[exponent-product = \cdot]{siunitx}
\sisetup{locale=DE}

\makeatletter
\renewcommand\@dotsep{5}
\makeatother

% Pakete für Grafiken
\usepackage{graphicx}
\usepackage{wrapfig}
\usepackage{epstopdf}
\usepackage{subcaption}
%\captionsetup[subfigure]{list=true, font=normalsize, labelformat=brace, position=top} %setup für subfigure captions
\usepackage{pstricks}
\usepackage{pst-plot,pst-bar} %Balkendiagramme

% Paket für Literaturverzeichnis
\usepackage[
    style=alphabetic,
    sorting=nty,
    sortcites=true,
    autopunct=true,
    autolang=hyphen,
    hyperref=true,
    abbreviate=false,
    backref=true,
    backend=biber,
    block=space
    ]{biblatex}

\addbibresource{bib/bib.bib} %Einfügen der Literaturbibliothek
\defbibheading{bibempty}{}


\usepackage{url}
\usepackage{hyperref}
\hypersetup{hidelinks}
\urlstyle{same}

%Abkürzungen durch Kommandos setzen
\newcommand{\bspw}{bspw.\xspace}
\newcommand{\bzw}{bzw.\xspace}
\newcommand{\etc}{etc.\xspace}
\newcommand{\zB}{z.\,B.\xspace}
\newcommand{\EV}{e.\,V.\xspace}
\newcommand{\zT}{z.\,T.\xspace}
\newcommand{\iVm}{i.\,V.\,m.\xspace}
\newcommand{\idR}{i.\,d.\,R.\xspace}
\newcommand{\ihv}{i.\,H.\,v.\xspace}
\newcommand{\ua}{u.\,a.\xspace}
\newcommand{\dH}{d.\,h.\xspace}
\newcommand{\vgl}{vgl.\xspace}
\newcommand{\ca}{ca.\xspace}
\newcommand{\dV}{d.\,Verf.}
\newcommand{\RNr}{Rn.\xspace}
\newcommand{\oa}{o.\,{ä}.\xspace}
\newcommand{\vC}{v.\,Chr.\xspace}
\newcommand{\nC}{n.\,Chr.\xspace}
\newcommand{\vA}{v.\,a.\xspace}

\usepackage{tocbibind} %damit Verzeichnisse im Inhaltsverzeichnis aufgeführt werden
%\usepackage[notoc]{tocbibind} %damit Inhaltsverzeichnis nicht im Inhaltsverzeichnis vorkommt

\pagenumbering{Roman} %Römische Seitennummerierung für Verzeichnisse

\begin{document}

%TITELSEITE
\begin{titlepage}
	\begin{center}
	\vspace*{0.2cm}
	
	\huge
	\textbf{Konzeption, Projektierung und Inbetriebnahme eines mehrachsigen Positioniersystems}
	
	\vspace*{2.0cm}
	\Large
	\textbf{Exposé zur Bachelorarbeit}
	
	\vspace*{1.2cm}
	\normalsize
	im Studiengang\\
	\Large
	Elektrotechnik
	
	\vspace*{0.9cm}
	\normalsize
	am Fachbereich\\
	\Large
	Ingenieurwissenschaften - Energie und Information
	
	\vspace*{0.9cm}
	\normalsize
	an der\\
	\Large
	Hochschule für Technik und Wirtschaft Berlin
	
	\vspace*{1.6cm}
	\normalsize
	vorgelegt von\\
	\Large
	Aaron Zielstorff
	
	\vspace*{0.7cm}
	\normalsize
	Berlin, 04.06.2021
	
	\vspace*{1.6cm}
	\normalsize
	Betreuer:\\
	Herr Prof. Dr. Stephan Schäfer\\
	Herr Dipl.-Ing. Dirk Schöttke
	
	\end{center}
\end{titlepage}


%INHALTSVERZEICHNIS
\setcounter{tocdepth}{2} %Inhaltsverzeichnis zeigt 2 Gliederungsebenen
\tableofcontents
\thispagestyle{empty}
\clearpage

%ABBILDUNGSVERZEICHNIS
%\listoffigures

%TABELLENVERZEICHNIS
%\listoftables
%\clearpage

\pagenumbering{arabic} %Arabische Seitennummerierung nach den ab Textbeginn

\section{Problemstellung}
Im Rahmen meiner Tätigkeiten als Praktikant an der Hochschule für Technik und Wirtschaft Berlin (im Folgenden abgekürzt mit HTW) ist über einen Zeitraum von einem Jahr ein Positioniersystem in einem Laborraum der Hochschule konstruiert worden. Dieses besteht aus zwei Achsen, und soll über einen Motioncontroller betrieben werden. Vor Beginn der Bearbeitungsperiode der Bachelorthesis konnte die Anlage physisch fertiggestellt werden.\\
Die entstehende Arbeit befasst sich mit der Konzeption eines Automatisierungssystemes, mit dessen Projektierung und der Inbetriebnahme, bei welchem es sich um eine mehrachsige Positioniereinheit handelt. Dabei soll die Anlage anforderungsorientiert konzipiert werden.

\section{Zielsetzung und Erkenntnisse}
Ziel der Bachelorarbeit ist es, ein Positioniersystem umzusetzen, welches von jedem Laborplatz aus genutzt werden kann, um verschiedenste anlagenspezifische Testszenarien zu erproben, wie beispielsweise das Fahren von Trajektorien. Dabei sollen aus dem Prozessablauf Daten zur Wertschöpfung generiert werden. Das bedeutet konkret, dass über einen oder mehrere Kommunikationskanäle von der Anlage aus Daten zur Verfügung gestellt werden, die extern weiterverarbeitet werden können.\\
\textbf{Ziele der Arbeit:}
\begin{compactitem}
	\item Analyse der Anforderungen an das Positioniersystem
	\item Auftsellen von Testkriterien für die ermittelten Anforderungen
	\item Die wichtigstens Anforderungen als Key Performance Indicators (KPI's) zusammentragen
	\item Entwickeln eines Konzeptes nach Handlungsempfehlungen bezüglich des Requierements Engineerings
	\item Darstellen der Anlagenprojektierung unter Nutzung von Modellen (UML)
	\item Implementieren der Modelle als Automatisierungssoftware mit dem \glqq Machine Expert Logic Builder\grqq{}
	\item Inbetriebnahme des Systems unter Prüfung der festgelegten Testkriterien
	\item Bereitstellen von Prozess- und Maschinendaten für die Weiterverarbeitung
	\item Bereitstellen von Schnittstellen für die Bedienung und Programmierung von jedem Laborplatz
	\item Mit Hilfe der Ergebnisse die folgende Problemfrage beantworten: \textit{Wie muss das Positioniersystem aufbereitet werden, um alle Anforderungen zu erfüllen?}
\end{compactitem}
Es wird erwartet, dass die Ergebnisse der Arbeit zeigen, dass mit Hilfe von Handlungsempfehlungen bezüglich des Requierements Engineerings (VDI) und der Anlagenprojektierung ein Positioniersystem aufbereitet werden kann, dass die Anforderungen an selbes erfüllt.\\
Dabei werden die Zugriffsmöglichkeit auf die Anlage von jedem Laborplatz und das Generieren sowie Bereitstellen von anlagenspezifischen Daten als KPI's ermittelt.\\
Die Inbetriebnahme am Ende der Arbeitsphase beinhaltet das erfolgreiche Verifizieren der Testkriterien nach eventuellen Verbesserungen der implementierten Modelle aus der Projektierungsphase.

\section{Forschungsstand und theoretische Grundlage}
Die Methodik des Requierements Engineerings wurde in der Literatur bereits behandelt und es existieren einige Quellen, die Handlungsempfehlungen zur Anwendung der Methodik beinhalten. Dennoch wird das Requierements Engineering noch zu selten angewendet, trotz der bekannten monetären und zeitlichen Vorteile \cite[xvii]{Laplante2014}. Die Methodik gilt als erfolgreiches Mittel, um mögliche Nachbearbeitungen von Software zu vermindern und die Kosteneffizienz zu verbessern \cite[1]{
Laplante2014}.\\
Grundlegend kann die Vorgehensweise in zwei Schritte unterteilt werden, die \glqq Artificial Development Sequence\grqq{} (künstliche Entwicklungssequenz) und die \glqq Engineering Activity Sequence\grqq{} (Aktivitätsentwicklungssequenz). Erste beschäftigt sich mit dem Aufstellen der Nutzer- und Systemanforderungen, sowie mit dem Systemdesign-Spezifikationen. Letztere beinhaltet das Testen der Anforderungen nach der Systementwicklung \cite[6]{
Laplante2014}.
Auch die Anlagenprojektierung ist in der Literatur bereits thematisiert worden und es existieren Empfehlungen für die Umsetzung dieser. Die Projektierung ist die Gesamtheit aller Entwurfs-, Planungs- und Koordiniriengsmaßnahmen, mit denen die Umsetzung eines Automatisierungsprojektes vorbereitet wird \cite[8]{Bindel2017}.\\
Die Umsetzung eines mehrachsigen Positioniersystems wurde in der Litaratur noch nicht behandelt. Diese Arbeit wird den Entwicklungsprozess vom Konzept bis zur Inbetriebnahme des Systems darstellen. 

\section{Arbeitskonzept}
Die folgenden Fragen sollen beantwortet werden:
\begin{compactitem}
	\item Nach welchem Vorgehen kann ein Automatisierungsprojekt umgesetzt werden?
	\item Welche Anforderungen existieren für das Positioniersystem?
	\item Wie muss ein mehrachsiges Positioniersystem Aufbereitet werden, um die Anforderungen zu erfüllen?
	\item Wie kann des Erfüllen der Anforderungen geprüft werden?
	\item Welche KPI's werden für das System erhoben?
	\item Wie können Prozessdaten zur Wertschöpfung bereitgestellt werden?
	\item Welche Schritte sind nötig, um die Anlage von jedem Laborplatz aus nutzen zu können?
	\item Wie müssen die Sensoren und Aktoren des Positioniersystems eingebunden werden?
	\item Welche Maßnahmen müssen für die Sicherheit von Mensch und Anlage getroffen werden?
	\item Ausblick: Welche Vorteile bietet die gewählte Aufbereitung des Positioniersystems in bezug auf die Thematik Digitalisierung und Industrie 4.0?
\end{compactitem}
Die Abschlussarbeit beinhaltet die Aufbereitung des mehrachsigen Positioniersystems von der Konzeption bis zur Inbetriebnahme. Die Konzeptionsphase beinhaltet die Analyse der Anforderungen der Stakeholder, das Aufstellen von KPI's und das Entwickeln von Testkriterien und Testmethoden für diese. Dabei wird ein besonderer Wert auf die Schnittstellen  des Systems gelegt, um zu gewährleisten, dass im Betrieb Prozessdaten der Positioniereinheit bereitgestellt werden, und diese von jedem Laborplatz aus Programmiert werden kann, um Beispielsweise Trajektorien Testen zu können.\\
Die Entwicklungsphase der Anlagenprojektierung beschäftigt sich mit der Modellierung der Systemfunktionen unter Nutzung der Unified Modeling Language (UML) und unter Berücksichtigung der IEC Norm 61131. Die entstandenen Modelle werden anschließend unter der von Schneider Electric zur verfügung gestellten Codesys-Entwicklungsumgebung \glqq Machine Expert Logic Builder\grqq{} implementiert.\\
Am Ende des Aufbereitungsprozesses steht die Inbetriebnahme des Positioniersystems, bei dem die Anforderungen an dieses mit Hilfe von Tests überprüft werden, und mögliche Fehler oder Versäumnisse ausgebessert werden. Als Resultat soll eine Anlage für den Laborbetrieb bereitstehen, die von den Studenten der HTW genutzt werden kann, um für das System typische Funktionen zu Testen und aus dem Betrieb Daten zu gewinnen, die in anderen Projekten weiterverarbeitet werden können und es ermöglichen, die Positioniereinheit in größere Strukturen einzubinden. Diese Strukturen sollen nur ausblickhaft erwähnt werden, da sie nicht Teil der Bachelorthesis sind, sondern sich in den Themenkomplex Industrie 4.0 eingliedern.

\section{Vorläufige Gliederung}
\begin{itemize}
	\item[1] Einleitung
	\item[2] Theoretische Grundlagen
	\begin{enumerate}[label*=\arabic*.]
		\item Requierements Engineering
		\item Anlagenprojektierung
	\end{enumerate}
	\item[3] Konzeption
	\begin{enumerate}[label*=\arabic*.]
		\item Vorstellung der Laboranlage
		\item Anforderungsanalyse
		\item Kontextanalyse
		\item Anwendungsfallspezifikation
		\item Verhaltensspezifikation
		\item Partitionierung
		\item Testspezifikation
	\end{enumerate}
	\item[4] Projektierung
	\begin{enumerate}[label*=\arabic*.]
		\item Genereller Aufbau der Automatisierungssoftware
		\item Implementierung der Modelle
		\item Peripherie Schnittstelle(n)
		\item Anwenderschnittstelle
	\end{enumerate}
	\item[5] Inbetriebnahme
	\begin{enumerate}[label*=\arabic*.]
		\item Implementierung
		\item Testen
		\item Überarbeitung/Verbesserungen
	\end{enumerate} 
	\item[6] Fazit
	\item[7] Ausblick
\end{itemize}

\section{Zeitplan}
\textbf{Dauer:} 18 Wochen (06.2021 - 09.2021)\\
Bis 30.04.: Literaturrecherche\\
Bis 20.05.: Thematische Hinführung + Hypothesen\\
Bis 20.06.: Rohfassung Modellierung des Systems (Konzeptteil)\\
Bis 04.07.: Fertigstellung der Automatisierungssoftware (Projektierung)\\
Bis 18.07.: Rohfassung Einleitung + Schluss\\
Bis 01.08.: Fertigstellen des physischen Baus des Positioniersystems
Bis 15.08.: Rohfassung der Inbetriebnahme (Nachtragungen Einleitung und Schluss)
Bis 29.08.: Überarbeitung und Korrektur\\
Bis 30.08.: Druck \\
Bis 31.08.: Abgabe\\
\textit{Zeitplan ist zum Ende hin eher ungenau aufgrund mangelnder Informationen bezüglich der terminlichen Umstände.}

\newpage
\section*{Literatur}
\addcontentsline{toc}{section}{Literatur}

\nocite{Laplante2014}
\nocite{Bindel2017}

\printbibliography[
	heading=subbibintoc,
	type=book,
	title={Bücher}
]
	
%\addcontentsline{toc}{section}{Literatur}

% \nocite{Wissenschaft2013}

% \printbibliography[
% 	heading=subbibintoc,
% 	type=article,
% 	title={Artikel}
% ]

\end{document}