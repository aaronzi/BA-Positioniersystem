\documentclass[12pt, a4paper, twoside]{article} %Dokumentenklasse Setzen

\usepackage[a4paper, left=2.5cm, right=2.5cm]{geometry} %Seitenrand setzen

\usepackage[T1]{fontenc}
\usepackage{lmodern}
\usepackage[utf8]{inputenc}
\usepackage[ngerman]{babel}
\usepackage{csquotes}
\usepackage{xpatch}
\usepackage{xspace} % setzten von Leerzeichen nach Abkürzungen
\usepackage{microtype} % für glättere Seitenränder

%Mathe Pakete
\usepackage{amsmath}
\usepackage{thmtools}
\usepackage{amsfonts}
\usepackage{amssymb}

%Listenumgebungen
\usepackage{listings}
\usepackage{paralist}
\usepackage{enumitem}

%Demo Text
\usepackage{blindtext}

% Farb-Pakete
\usepackage{xcolor}
\usepackage{colortbl}

% Für erweiterte Tabellen
\usepackage{longtable}

% Für Einheiten
\usepackage[exponent-product = \cdot]{siunitx}
\sisetup{locale=DE}

\makeatletter
\renewcommand\@dotsep{5}
\makeatother

% Pakete für Grafiken
\usepackage{graphicx}
\usepackage{wrapfig}
\usepackage{epstopdf}
\usepackage{subcaption}
%\captionsetup[subfigure]{list=true, font=normalsize, labelformat=brace, position=top} %setup für subfigure captions
\usepackage{pstricks}
\usepackage{pst-plot,pst-bar} %Balkendiagramme

% Paket für Literaturverzeichnis
\usepackage[
    style=alphabetic,
    sorting=nty,
    sortcites=true,
    autopunct=true,
    autolang=hyphen,
    hyperref=true,
    abbreviate=false,
    backref=true,
    backend=biber,
    block=space
    ]{biblatex}

\addbibresource{bib/bib.bib} %Einfügen der Literaturbibliothek
\defbibheading{bibempty}{}


\usepackage{url}
\usepackage{hyperref}
\hypersetup{hidelinks}
\urlstyle{same}

%Abkürzungen durch Kommandos setzen
\newcommand{\bspw}{bspw.\xspace}
\newcommand{\bzw}{bzw.\xspace}
\newcommand{\etc}{etc.\xspace}
\newcommand{\zB}{z.\,B.\xspace}
\newcommand{\EV}{e.\,V.\xspace}
\newcommand{\zT}{z.\,T.\xspace}
\newcommand{\iVm}{i.\,V.\,m.\xspace}
\newcommand{\idR}{i.\,d.\,R.\xspace}
\newcommand{\ihv}{i.\,H.\,v.\xspace}
\newcommand{\ua}{u.\,a.\xspace}
\newcommand{\dH}{d.\,h.\xspace}
\newcommand{\vgl}{vgl.\xspace}
\newcommand{\ca}{ca.\xspace}
\newcommand{\dV}{d.\,Verf.}
\newcommand{\RNr}{Rn.\xspace}
\newcommand{\oa}{o.\,{ä}.\xspace}
\newcommand{\vC}{v.\,Chr.\xspace}
\newcommand{\nC}{n.\,Chr.\xspace}
\newcommand{\vA}{v.\,a.\xspace}

\usepackage{tocbibind} %damit Verzeichnisse im Inhaltsverzeichnis aufgeführt werden
%\usepackage[notoc]{tocbibind} %damit Inhaltsverzeichnis nicht im Inhaltsverzeichnis vorkommt

\pagenumbering{Roman} %Römische Seitennummerierung für Verzeichnisse

\begin{document}

%TITELSEITE
\begin{titlepage}
	\begin{center}
	\vspace*{0.2cm}
	
	\huge
	\textbf{Konzeption, Projektierung und Inbetriebnahme eines mehrachsigen Positioniersystems}
	
	\vspace*{2.0cm}
	\Large
	\textbf{Exposé zur Bachelorarbeit}
	
	\vspace*{1.2cm}
	\normalsize
	im Studiengang\\
	\Large
	Elektrotechnik
	
	\vspace*{0.9cm}
	\normalsize
	am Fachbereich\\
	\Large
	Ingenieurwissenschaften - Energie und Information
	
	\vspace*{0.9cm}
	\normalsize
	an der\\
	\Large
	Hochschule für Technik und Wirtschaft Berlin
	
	\vspace*{1.6cm}
	\normalsize
	vorgelegt von\\
	\Large
	Aaron Zielstorff
	
	\vspace*{0.7cm}
	\normalsize
	Berlin, 04.06.2021
	
	\vspace*{1.6cm}
	\normalsize
	Betreuer:\\
	Herr Prof. Dr. Stephan Schäfer\\
	Herr Dipl.-Ing. Dirk Schöttke
	
	\end{center}
\end{titlepage}


%INHALTSVERZEICHNIS
\setcounter{tocdepth}{2} %Inhaltsverzeichnis zeigt 2 Gliederungsebenen
\tableofcontents
\thispagestyle{empty}
\clearpage

%ABBILDUNGSVERZEICHNIS
%\listoffigures

%TABELLENVERZEICHNIS
%\listoftables
%\clearpage

\pagenumbering{arabic} %Arabische Seitennummerierung nach den ab Textbeginn

\section{Problemstellung}
Seit einigen Jahren zeichnet sich eine vierte industrielle Revolution ab. Hervorgerufen durch das Voranschreiten der Digitalisierung und der damit einhergehenden Vernetzung auf der einen Seite, und der immer größer werdende Nachfrage nach Personalisierung, Effizienz und Qualität industrieller Produkte auf der anderen Seite, findet ein globaler Paradigmenwechsel statt \cite[S. 33]{Bauernhansl2014} Für Deutschland stellt diese Entwicklung eine große Chance dar, um die industrielle Produktion ausbauen zu können \cite[S. 1]{Pistorius2020}.\\
Im Zentrum der Digitalisierung Industrieller Systeme steht die Wandelbarkeit von Fabriken und deren Anlagen (Smart Applications), die Vorhersage von Wartungsnotwenigkeiten (Predictive Maintenance), die Kosten- und Energieoptimierung, sowie die Monetarisierung von gewonnenen Daten. Entscheidend dabei ist die industrielle Kommunikation, die sich durch den Datenaustausch über internetbasierte Kommunikationsstandards auszeichnet. Als Basisentwicklung für diese Anforderungen gelten Cyber-physische Systeme (CPS).

\section{Zielsetzung und Erkenntnisse}
Ziel dieser Bachelorarbeit ist es, den Konzeptionsprozess, die Projektierung und Inbetriebnahme einer Automatisierungssoftware am Beispiel eines mehrachsigen Positioniersystems darzustellen. Dabei wird das Positioniersystem als bereits physisch/elektrisch einsatzbereit angenommen. Es gilt im Bezug auf seine Hardware als vollständig bestückt mit allen notwendigen Komponenten, die für die Umsetzung eines CPS notwenig sind. \\
\textbf{Ziele der Arbeit:}
\begin{compactitem}
	\item Klärung der Begriffe Cyber-physisches System und Industrie 4.0
	\item Definition von Aufgaben und Ableitung der Anforderungen an das Positioniersystem und dessen Software
	\item Entwickeln von Testkriterien/Testsystemen für die definierten Anforderungen
	\item Darstellen des Konzeptionsprozesses der Automatisierungssoftware 
	\item Projektierung und Modellierung der Systemsoftware für das mehrachsige Positioniersystem
	\item Inbetriebnahme des Systems unter Prüfung der festgelegten Testkriterien
	\item Bereitstellen von Prozess- und Maschinendaten für die Weiterverarbeitung und Monetarisierung dieser
	\item Mit Hilfe der Ergebnisse die folgende Forschungsfrage beantworten: \textit{Wie sollte eine Automatisierungssoftware für ein mehrachsiges Positionsystems konzipiert, projektiert und in Betrieb genommen werden, um das Positioniersystem als Cyber-physisches System nach dem Leitbild Industrie 4.0 qualifizieren zu können?}
\end{compactitem}
Es wird erwartet, dass die Ergebnisse der Forschung zeigen, dass die Umsetzung der Automatisierungssoftware unter den dem Leitbild Industrie 4.0 die Grundprinzipien von CPS im Beispiel des mehrachsigen Positioniersystems widerspiegeln. Weiterhin ist damit zu rechnen, dass die Projektierung analog auch bei anderen Industrieystemen und Anlagen nach selbem Leitbild umzusetzen ist. Außerdem steht in Aussicht, dass durch das Bereitstellen von Prozessdaten sowohl ein finanzieller als auch qualitativer Mehrwert durch das CPS gewonnen werden kann. 

\section{Forschungsstand und theoretische Grundlage}
Der Begriff Industrie 4.0 ist im Rahmen eines Zukunftprojekts der deutschen Bundesregierung entstanden. Durch die Digitalisierung klassischer Industrieunternehmen wird auf eine Steigerung der Automatisierung und Vernetzung in der Produktion abgezielt, um die Wettbewerbsfähigkeit auf dem globalen Markt sicherstellen zu können \cite[S. 63]{Winkelhake2021}. Cyber-physische Systeme (CPS) gelten als Basisinnovation für die vierte industrielle Revolution. In Zukunft werden Unternehmen ihre Einrichtungen, Anlagen, Maschinen und Betriebsmittel mithilfe von CPS global vernetzen \cite[S. 5]{Wissenschaft2013}. CPS können definiert werden als eingebettete Systeme, die
\begin{compactitem}
	\item durch Unterstützung von Sensoren physikalische Daten generieren und mittels Aktoren reale Vorgänge beeinflussen,
	\item Daten sichern als auch verarbeiten und daraus Handlungen ableiten,
	\item über Kommunikationsschnittstellen untereinander verbunden sind, egal ob lokal oder global sowie drahtlos oder drahtgebunden,
	\item bereitstehende Dienste und Daten ortsunabhängig nutzen und anbieten,
	\item unterschiedliche Möglichkeiten zur Kommunikation und Steuerung in Form von
Mensch-Maschine-Schnittstellen zur Verfügung stellen \cite[S. 22]{Geisberger2012}.
\end{compactitem}
Wie die Software für ein mehrachsiges Positioniersystem konzipiert, Projektiert und in Betrieb genommen wird, wurde in der Literatur noch nicht behandelt.
%Der Hauptteil der Bachelorarbeit, die Konzeption, Projektierung und Inbetriebnahme eines mehrachsigen Positionsystems wird in der Literatur behandelt, dient dieser Arbeit jedoch nicht als Grundlage. Es existieren vergleichbare Anlagen größerer Dimension in der Logistikbranche. Ein Beispiel wäre das automatisierte Hochregallager, welches unter anderem in den Logistikzentren von Amazon anzutreffen ist.

\section{Forschungskonzept}
Die folgenden Fragen sollen beantwortet werden:
\begin{compactitem}
	\item Warum sollten Industrielle Anlagen digitalisiert werden?
	\item Erfüllt die projektierte Anlage unter Ausführung der entwickelten Software die Definitionskriterien eines CPS?
	\item WWie muss eine Automatisierungssoftware modelliert werden, die es ermöglicht Daten zur weiterverarbeitung bereitzustellen?
	\item Wie müssen die Sensoren und Aktoren des Positioniersystems eingebunden werden, um die Anforderungen aus dem Konzept zu erfüllen?
	\item Wie können die systemrelevanten Daten für die externe Verarbeitung bereitgestellt werden?
	\item Welche Maßnahmen müssen für die Sicherheit von Mensch und Anlage getroffen werden?
	\item Wie kann für einen sicheren Datenaustausch zwischen Anlage und Peripherie gesorgt werden?
	\item Welche Vorteile ergeben sich durch die Entwicklung einer Software, welche das System zu einem CPS werden lässt?
	\item Ausblick: Welche weiterführenden Möglichkeiten und Funktionen kann das Positioniersystem bereitstellen?
\end{compactitem}
Die Literaturarbeit umfasst eine Zusammenfassung des Konzeptes Industrie 4.0 und die Definition eines Leitbildes für CPS.
Dabei soll insbesondere auch auf das industrielle Internet der Dinge (IIoT) eingegangen werden. Aus dieser Analyse wird anhand des Beispiels eines mehrachsigen Positioniersystems der Entwicklungsprozess einer Automatisierungssoftware dargestellt.\\
Die Forschungsarbeit beinhaltet die Modellierung dieser Automatisierungssoftware von der Konzeption bis zur Inbetriebnahme in einem Hochschul-Labor. Die Konzeptionsphase betrachtet den Prozess von der Analyse der Systemaufgaben, über die Definition der Anforderungen, bis zur Festlegung von Testkriterien für letztere. Dabei werden die Use-Cases untersucht, die Anforderungen aller das Positioniersystem betreffenden Stakeholder zusammengetragen und Schnittstellen des System berücksichtigt.\\
Hauptteil der Bachelorarbeit soll es sein den Softwareentwicklungsprozess von der Idee bis zum Betrieb darzustellen. Dabei soll weniger auf die konstruktionelle Natur des mehrachsigen Positioniersystem eingegangen werden, sondern auf die modellierung der Automatisierungssoftware. Als Ergebnis soll eine Handlungsempfehlung vorliegen, die aufzeigen soll, welche Schritte bei der Entwicklung der Software durchlaufen werden und welche Anforderungen erfüllt werden müssen, um eine Anlage als CPS nach dem Leitbild Industrie 4.0 zu Qualifizieren, Daten bereitzustellen, die in einer fortführenden Arbeit weiterverarbeitet und Monetarisiert werden können.\\
Abschließend soll die Arbeit einen Ausblick bieten, wie die gewonnenen Daten weiterverarbeitet, monetarisiert und zur Optimierung der Anlage genutzt werden können.

% Festlegung der Funktionalität, welche Use-Cases behandelt werden und wie das System dem Leitbild Industrie 4.0 folgen kann. In der Projektierung wird der Umsetzungsprozess konkretisiert, indem die benötigten Komponenten ermittelt und dimensioniert werden. Darauf folgt die Planung der Energieversorgung, welche gefolgt wird von der Montage des Mehrachssystems. Vor der Inbetriebnahme muss die Sicherheit des Systems mit besonderer Priorität behandelt werden und verlangt eine eigenständige Planung, Umsetzung und Integration in das Positionsystem. Die Inbetriebnahme zeichnet sich durch das Etablieren der Kommunikation zwischen den Komponenten und möglichen Schnittstellen zu externen Systemen aus. Weiterhin verlangt die Nutzung des Achssystems eine Programmierung, die Anfangs noch rudimentär ist, später jedoch beliebig erweitert werden kann, um Aspekte der einer Industrie 4.0 Umgebung zu testen und demonstrieren. 

\section{Vorläufige Gliederung}
\begin{itemize}
	\item[1] Einleitung
	\item[2] Theoretische Grundlagen
	\begin{enumerate}[label*=\arabic*.]
		\item Industrie 4.0
		\item CPS
		\item IIoT
	\end{enumerate}
	\item[3] Konzeption
	\begin{enumerate}[label*=\arabic*.]
		\item Vorstellung der Laboranlage
		\item Anforderungsanalyse
		\item Kontextanalyse
		\item Anwendungsfallspezifikation
		\item Verhaltensspezifikation
		\item Partitionierung
		\item Testspezifikation
	\end{enumerate}
	\item[4] Projektierung
	\begin{enumerate}[label*=\arabic*.]
		\item Genereller Aufbau der Software
		\item Implementierung der Modelle
		\item Peripherie Schnittstelle(n)
		\item Anwenderschnittstelle
	\end{enumerate}
	\item[5] Inbetriebnahme
	\begin{enumerate}[label*=\arabic*.]
		\item Implementierung
		\item Testen
		\item Überarbeitung/Verbesserungen
	\end{enumerate} 
	\item[6] Fazit
	\item[7] Ausblick
\end{itemize}

\section{Zeitplan}
\textbf{Dauer:} 18 Wochen (06.2021 - 09.2021)\\
Bis 30.04.: Literaturrecherche\\
Bis 20.05.: Thematische Hinführung + Hypothesen\\
Bis 20.06.: Rohfassung Modellierung des Systems (Konzeptteil)\\
Bis 04.07.: Fertigstellung der Software (Projektierung)\\
Bis 18.07.: Rohfassung Einleitung + Schluss\\
Bis 01.08.: Fertigstellen des physischen Baus des Positioniersystems
Bis 15.08.: Rohfassung der Inbetriebnahme (Nachtragungen Einleitung und Schluss)
Bis 29.08.: Überarbeitung und Korrektur\\
Bis 30.08.: Druck \\
Bis 31.04.: Abgabe\\
\textit{Zeitplan ist zum Ende hin eher ungenau aufgrund mangelnder Informationen bezüglich der terminlichen Umstände.}

\newpage
\section*{Literatur}
\addcontentsline{toc}{section}{Literatur}

\nocite{Winkelhake2021}
\nocite{Geisberger2012}
\nocite{Bauernhansl2014}
\nocite{Pistorius2020}

\printbibliography[
	heading=subbibintoc,
	type=book,
	title={Bücher}
]
	
%\addcontentsline{toc}{section}{Literatur}

\nocite{Wissenschaft2013}

\printbibliography[
	heading=subbibintoc,
	type=article,
	title={Artikel}
]

\end{document}