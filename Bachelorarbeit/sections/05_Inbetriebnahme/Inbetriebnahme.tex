\documentclass[../../Bachelorarbeit.tex]{subfiles}
\begin{document}

\section{Inbetriebnahme}
Dieses Kapitel unterteilt sich in drei Abschnitte. Das erste Unterkapitel behandelt die Implementation \bzw die Implementationsphase der Realisierung des mehrachsigen Positioniersystems. Dabei geht es auf der einen Seite um die Beschreibung des Hardware- \bzw Geräteentwurfs aus den zugehörigen Inhalten im Kapitel zur Projektierung der Anlage. Konkret meint dies den konstruktionellen Aufbau der Laboranlage aus der Konfigurator Grafik und der Konstruktionszeichnung zum einen, zum anderen die elektrotechnische Implementierung des entwickelten Stromlaufplanes. Auf der anderen Seite beinhaltet das sich anschließende Kapitel die Implementation der Software aus den in der Modellierung erstellten UML Diagrammen wie unter anderem das Zustandsdiagramm.\\
Nach der Implementation kann und muss das System auf seine Funktionalität geprüft werden, sowohl was seine Hardware als auch seine Software anbelangt. Dazu werden die Testkriterien aus der Anforderungsanalyse herangezogen und in Testfällen zur Überprüfung aller Anforderungen genutzt. Dabei wird eine Unterteilung in mehrere Ebenen vorgenommen, die aufbauend nacheinander durchgangen werden. Ziel ist es zunächst grundlegende Testfälle zu behandeln, die Voraussetzung für speziellere, untergeordnete Validierungen von Testspezifikationen sind. \\
Wird ein Test nicht bestanden, muss in einer weiteren Iteration der Entwicklungsphase der Mangel beseitigt werden. Diese Korrekturen werden im letzten Unterkapitel der Inbetriebnahme behandelt. 


\end{document}