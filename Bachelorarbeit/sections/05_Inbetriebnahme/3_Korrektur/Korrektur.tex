\documentclass[../../../Bachelorarbeit.tex]{subfiles}
\begin{document}

\subsection{Korrekturen und Verbesserungen}
% \color{red}
% Hier sollten die nicht erfüllten Testfälle noch kurz diskutiert werden.\\
% Getriebe x-Achse, Nichtverdrahtung der Bedienelemente an Schaltschrank, Netzdrossel und Netzschütz
Dieses Kapitel befasst sich mit der Aufarbeitung der Testergebnisse aus dem letzten Unterabschnitt. Die meisten Testfälle konnten erfolgreich erfüllt werden und das mehrachsige Positioniersystem ist grundsätzlich für den aktiven Betrieb einsatzbereit. Die Kernanforderungen wurden bestanden.\\
Dennoch sind vereinzelt Testfälle noch nicht verifiziert worden oder haben die Prüfung sogar nicht bestanden. Im Folgenden sollen diese nun kurz diskutiert werden, um mögliche Lösungsansätze zu entwickeln, damit das System entsprechend seiner Anforderungen vollständig in Betrieb genommen werden kann.\\
Bereits bei der Sichtprüfung fällt direkt auf, dass die durch den Prozessentwickler geforderte Greiffunktion schon auf Hardwareebene nicht implementiert wurde. Da es sich um einen möglichen Anwendungsfall des Systems handelt und keine grundlegende Anforderung der wichtigen Stakeholder (Lehrpersonal) nicht erfüllen lässt, sind zunächst keine weiteren Handlungen nötig. Die Funktionalität auf sowohl der Hardware- als auch Softwareebene kann als spätere Erweiterung kategorisiert werden, die nicht mehr Teil dieses Projektes ist.\\
Bei der Sichtprüfung fällt des weiteren das Fehlen von Plexiglasscheiben am linken und rechten Ende des Gehäuses auf. Damit das System in Anwesenheit von Personen genutzt werden darf, müssen diese unabdinglich am Gestell des Systems montiert sein. Grund für das Fehlen ist eine Lieferschwierigkeit. Folglich muss die Lieferung abgewartet und die Montage nachgeholt werden. Es sind keine weiteren Handlungen von Nöten.\\
Abschließend im Sichtprüfungstestfall konnte das Nichtvorhandensein eines Netzschützes und einer Netzdrossel festgestellt werden. Beide Bauteile sind für die Nutzung des Systems nicht zwangsläufig nötig, sorgen jedoch für eine Qualitätssteigerung der Anlage hinsichtlich Komfort und Netzverträglichkeit. Wie schon im vorherigen Punkt ist das Fehlen der Komponenten beding durch Lieferschwierigkeiten. Liegen die Komponenten zu einem späteren Zeitpunkt vor, müssen sie lediglich noch montiert und verdrahtet werden.\\
Bei der Prüfung der Verdrahtung fällt ein weiteres Nichtbestehen eines Testfalls auf. Die Bedienelemente an der Tür des Schaltschrankes sind nicht verdrahtet. Sämtliche Eingaben, die über das physische Bedienfeld stattfinden, können auch über die \acs{gui} des Systems vorgenommen werden. Es gilt die Verdrahtung nachzuholen.\\
Bei der Testung der Anlagenfunktionen können abseits der bekannten Fehler noch drei weitere Probleme festgestellt werden. Zunächst muss der Automatikmodus abschließend getestet werden. Das heißt konkret, dass die Trajektorievorgabefunktion und das absolute Positionieren der Achsen über das genutzte \textit{MotionTemplateFull} Programm, verifiziert werden muss. Weiterhin sind Geschwindigkeitsanpassungen der Achsbewegungen noch nicht dynamisch implemetiert worden. Grund dafür ist die fehlende Verdrahtung der vorgesehenen Potentiometer. Somit ergibt sich die Notwendigkeit einer Programmanpassung zum Regeln der Geschwindigkeit (und Beschleunigung) aller Bewegungen der Positioniereinheit. Dies schließt das gedämpfte Fahrverhalten in Endlagennähe ein. Zuletzt ist es erforderlich die funktionale Sicherheit sowie die OPC Kommunikation vollständig zu überprüfen, da einige Tests noch nicht vorgenommen wurden.\\
Bei der Nutzung des Systems ist ein nicht direkt über einen Testfall abgedecktes Problem zu erkennen. Das Positionieren auf der x-Achse des Systems ist möglich, jedoch ist das Motomoment nicht ausreichend, um die Achse flüssig bewegen zu können. Lösungsstrategie ist es den Servomotor nachträglich um ein getriebe zu erweitern, damit dieser die x-Achse zuverlässig bewegen kann. Benötigt wird eine Getriebe mit Übersetzungsverhältnis von 1:3 oder höher. Der Einbau macht eine Umkonfigurierung der mechanischen Parameter in den Servoreglereinstellungen erforderlich.\\
\bigskip \newline
Nach der Implementation der Korrekturen an der Systemhard- und Software gilt die Laboranlage als fertiggestellt. 

\end{document}