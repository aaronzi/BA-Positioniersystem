\documentclass[../Bachelorarbeit.tex]{subfiles}
\begin{document}

\section{Theoretische Grundlagen}
In diesem ersten Kapitel des Hauptteils der Arbeit sollen zunächst die theoretischen Aspekte als Grundlage für die folgenden Kapitel beleuchtet und erleutert werden. Ohne an dieser Stelle tiefer in Details einzudringen, ist das grundsätzliche Ziel dieser Arbeit, ein mehrachsiges Positioniersystem zu entwickeln und realisieren, so dass es im Laborbetrieb von Studierenden genutzt werden kann. Der Prozess der Umsetzung des Systems wird in dieser Arbeit in drei wesentliche Schritte unterteilt, den \textbf{Konzeptionsteil}, den \textbf{Projektierungsteil} und die \textbf{Inbetriebnahme}. Für die erfolgreiche und normkonforme Dokumentation und Nutzung der drei Kapitel bis hin zur Inbetriebnahme des positioniersystems als Laboranlage an der \ac{htw} Berlin, muss zunächst ein Grundgerüst gesetzt werden. Dieses besteht vor allem aus den beiden essenziellen Themen \textbf{Requierements Engineering} (zu deutsch Anforderungsentwicklung oder auch Anforderungsmanagement) und Anlagenprojektierung mit dem Fokus auf Automatisierungsanlagen.\\
Die Konzeptionsphase ist besonders geprägt durch das Requierements Engineering. Insbesondere der Abschnitt der Anforderungsanalyse unter Einbeziehung der Stakeholder und der Testspezifikation(en) zu den Anforderungen, begründen sich auf diesem. Die nachfolgende Systemanalyse im Konzeptionsteil der Arbeit basiert auf den Anforderungen, die mit Hilfe der Dokumentationsempfehlungen des Requierements Engineerings entstanden sind.\\
Vor allem der zweite Teil dieser Arbeit, die Projektierung bedarf der Behandlung von theoretischen Grundlagen zur Projektierung von Automatisierungsanlagen. Bereits die Analysephase in der Systemkonzeption kann der Projektierung zugeornet werden. In dieser Arbeit liegt der Schwerpunkt im Kapitel der Projektierung jedoch insbesondere auf der Entwicklung der Automatisierungssoftware, bei der die Modelle (vor allem \acs{uml} Diagramme) aus dem Konzeptionsteil implementiert werden. Für die Anlagenprogrammierung bedarf es der Einhaltung der gültigen Norm DIN/EN 61131, die auf der IEC Norm 61131 basiert uns sich mit speicherprogrammierbaren Steuerungen befasst. Insbesondere Teil 3 der Norm beschäftigt sich mit den Programmiersprachen, die auch in der auf Codesys basierenden Entwicklungumgebung, welche in dieser Arbeit genutzt wird, zur verwendung kommen. \\
Der letzte Teil der Arbeit, die Inbetriebnahme, stellt die Verifizierung der Anforderungen aus dem Konzeptionsteil dar. Es handelt sich also um das logische Endstück für das Vorgehen nach den Handlungsempfehlungen des Requierements Engineering. Endstück meint jedoch nicht, dass ein System als fertig umgesetzt gilt, wenn es in der Phase der Inbetriebnahme angelangt ist. Bei der Anforderungsanalyse handelt es sich um ein iteratives Verfahren, bei dem aus den Ergebnissen der vorherigen Iteration mögliche neue Anforderungen entstehen könnten.\\ % Quellen einbauen
In dem sich anschließenden Abschnitt wird zunächst das Requierements Engineering als theoretische Grundlage erleutert. Daran schließt sich nachfolgend die Erklkärung zur Vorgehensweise in der Anlagenprojektierung für Automatisierungsanlagen an. 

\subsection{Requierements Engineering}
Das \ac{re} ist der Zweig der Ingenieurwissenschaften, der sich mit den realen Zielen, Funktionen und Beschränkungen von Systemen befasst. Weiterhin untersucht es die Beziehung zwischen den genannten Faktoren, um die Spezifikationen des Systemverhaltens zu präzisieren, sowie deren Entwicklung im Laufe der Zeit und über Familien von verwandten Systemen hinweg darzustellen \cite[2-3]{Laplante2014}. Ziel der Anwendung des Requierements Engineerings ist zum einen das Verhindern von unnötigen Aufwänden im Verlauf des Lebenszyklus eines Systems. Außerdem kann durch die Einhaltung von Handlungsempfehlungen aus dem \acs{re} die Kosteneffektivität optimiert werden.

\end{document}