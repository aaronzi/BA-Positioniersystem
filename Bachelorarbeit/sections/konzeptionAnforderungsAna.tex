\documentclass[../Bachelorarbeit.tex]{subfiles}
\begin{document}

\subsection{Anforderungsanalyse}
In der Analysephase der Systementwicklung werden die Kundenanforderungen zusammengetragen und untersucht. %TODO: Quelle
Dabei stellt die Anforderungsanalysephase den ersten Schritt zum Aufstellen der initialen Dokumente für den Prozess dar. In weiteren Iterationen liegen der Anforderungsanalyse zusätzlich zu der ursprünglichen Aufgabenstellung noch die Ergebnisse der Tests und die erkannten Analysefehler ebenfalls als Quelle vor.\\
Die Ermittelten Anforderungen werden untergliedert in funktionale und nicht-funktionale Anforderungen. Diese Unterteilung findet in der Arbeit in separaten Unterkapiteln statt, die sich nachfolgend anschließen. Die Identifikation der Stakeholder ist grundsätzlich der Anforderungsanalyse zugehörig, wird jedoch in einem gesonderten, sich der Anforderungsanalyse anschließenden, Kapitel behandelt, da es sich im Kontext dieser Arbeit um ein Kernabschnitt handelt.\\
% TODO: Grafik Analysephase einbinden
Zur übersichtlichen Einordnung des jeweiligen Analyseschrittes wird die Grafik Analysephase eingeführt, an der sich die fogenden Kapitel entlangbewegen. Die Anforderungsanalyse kann auf der linken Seite der Grafik identifitziert werden und untergliedert sich in die bereits erwähnten drei Unterpunkte.\\
Die folgenden Abschnitte betrachten die Erstellung einer konkreten Anforderungsspezifikation, die zum Startbeginn des Entwicklungsprozesses vorliegen muss. In den Kapiteln zu den funktionalen und nicht-funktionalen Anforderungen werden die notwendigen Anforderungen für die Entwicklung der Laboranlage vorgestellt. Dabei sind die Hardwareanforderungen nur Beispielhaft aufgelistet. Eine komplette Liste der Anforderungen kann im Anhang gefunden werden. \\
% TODO: Quelle einfügen
Aus der theoretischen Grundlagen bereits erkenntlich, bestehen Anforderungen aus Zielen, die im Rahmen der Entwicklung erreicht werden sollen. Dabei handelt es sich um einfachen Text, der nach Absprachen mit dem Kunden Dokumentiert wurde. Konkret geht es im Fall dieser Arbeit um die definierten Aufgaben und Ziele, welche durch Professoren/innen und Laboringenieure/innen des Fachbereiches dokumentiert wurden. Im ersten Schritt ist es notwendig die Aufgaben zu konkretisieren, um überflüssige und irrelevante Lösungen die Aufgaben betreffend zu vermeiden.\\
Ausgangspunkt für die Entwicklung des mehrachsigen Positioniersystems sind folgende Anforderungen \bzw Ziele. Es wird gefordert, eine Laboranlage zu entwerfen, die simple Transportgüter sicher von einem Ablagepunkt zu einem anderen Ablagepunkt transportieren kann. Dies soll über zunächst zwei Achsen geschehen, die es ermöglichen Bewegungen in horizontale Richtung (X-Achse) und vertikale Richtung (Z-Achse) durchzuführen. Dabei ist es relavant, dass verschiedene Trajektorien von der Anlage gefahren werden können, welche durch den Nutzer programmatisch vorgegeben werden. Die Bewegung der Achsen erfolgt über zwei getrennt ansteuerbare Servomotoren, die über einen Servoantrieb mit einer Industriesteuerung verbunden sind. Die Steuerungskomponenten sind bereits vorhanden. Konkret handelt es sich um den LMC101 (Logic Motion Controller) von Schneider Electric, das LXM 62P Powersupply (ebenfalls von Schneider Electric) und den LXM 62D Double Drive. Zusätzlich soll eine PFC200 Steuerung von Wago zum einsatz kommen, mit der Betriebsströme gemessen und für die Weiterverarbeitung bereit gestellt werden können. Weiterhin sollen auch Prozessdaten aus dem Programmablauf des LMC101 für die externe Verarbeitung zur Verfügung stehen. Es ist vorgegeben, dass diese Daten per OPC Schnittstelle ausgelesen werden können. Kernziel bei der Entwicklung des Laborsystems ist es die Möglichkeit bereitzustellen, dass die Positioniereinheit von jedem Laborplatz programmiert und als Testsystem für den Lehrzweck eingesetzt werden kann. Für den Betrieb der Anlage sind zwei Betriebsmodi vorgesehen. Ersterer, der Automatikbetrieb soll einen Vollautomatischen Prozessablauf ermöglichen, bei welchem eine konkrete Positionieraufgabe zyklisch durchgeführt wird. Zweiterer, der Handbetrieb, nimmt manuelle Steuerbefehle vom Nutzer entgegen, bei welchen über Tastereingaben an der Laboranlage, Fahrbewegungen entlang der beiden Achsen durchzuführt werden. Ein Wechsel der Betriebsmodi ist über einen Wahltaster zu implementieren. Außerdem ist ein Schutz für die Anlage und deren Nutzer, sowie sich um das Positioniersystem befindende Personen vorgesehen. Der Schutz ist manuell auslösbar über Not-Halt Taster an der Laboranlage und durch ein Lichtvorhang vor dem Fahrbereich der beiden Achsen. Es soll zu einem späteren Zeitpunkt noch Möglich sein das System um weitere Achsen und Peripheriegeräte wie beispielsweise Förderbänder zu erweitern.

\subsubsection{Funktionale Anforderungen}


\subsubsection{Nicht-funktionale Anforderungen}

\end{document}