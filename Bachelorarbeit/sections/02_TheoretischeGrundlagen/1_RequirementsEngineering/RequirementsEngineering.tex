\documentclass[../../../Bachelorarbeit.tex]{subfiles}
\begin{document} 

\subsection{Requierements Engineering}
Das \ac{re} ist der Zweig der Ingenieurwissenschaften, der sich mit den realen Zielen, Funktionen und Beschränkungen von Systemen befasst. Weiterhin untersucht es die Beziehung zwischen den genannten Faktoren, um die Spezifikationen des Systemverhaltens zu präzisieren, sowie deren Entwicklung im Laufe der Zeit und über Familien von verwandten Systemen hinweg darzustellen \cite[2-3]{Laplante2014}. Ziel der Anwendung des Requierements Engineerings ist zum einen das Verhindern von unnötigen Aufwänden im Verlauf des Lebenszyklus eines Systems. Außerdem kann durch die Einhaltung von Handlungsempfehlungen aus dem \acs{re} die Kosteneffektivität optimiert werden.

\end{document}