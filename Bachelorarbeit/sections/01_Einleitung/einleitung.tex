\documentclass[../../Bachelorarbeit.tex]{subfiles}
\begin{document}

\section{Einleitung}
% \color{red}
% Einleitung fehlt und muss unbedingt vor Abgabe noch hinzugefügt werden!!!\\

% \bigskip
% Einige Quellen sind noch auskommentiert und fehlen teilweise sogar im Verzeichnis. Verlinkung hinzufügen, dass alle Quellen richtig angezeigt werden!!!

Die heutige Zeit ist geprägt durch die Digitalisierung der Industrie und die Automatisierung sowohl industrieller als auch alltäglicher Prozesse. Geleitet durch die Automatisierungstechnik, die in der zweiten Hälfte des letzten Jahrhunderts aufkam, beeinflussen technische Systeme mittlerweile Großteile unseres Lebens \cite[10]{Abolhassan2016}.\\
Mit der Erfindung des Transistors in den 60er-Jahren kam es zum Boom der Computertechnik. So auch wurde die wichtigste Komponente für moderne Industrieanlagen zur Realität. Die \ac{sps}, ein programmierbares Steuerelement, ermöglichte es zu jedem Zeitpunkt das Verhalten eines Prozesses vorzugeben. Bislang waren verbindungsprogrammierbare Steuerungen (VPSen), die fest verdrahtet waren, Stand der Entwicklung. Nunmehr konnten ganze Fabriken mit der gleichen Steuerung betrieben werden \cite[326]{Heinrich2019}.\\
Durch den Einzug der \acs{sps} in die Industrie wurde es erforderlich, Computerprogramme zu schreiben, um Prozesse zu automatisieren. Für die Entwicklung der Software solcher Systeme wurde die Norm IEC 61131 verabschiedet, die als Richtlinie zur \acs{sps} Programmierung zu verstehen ist \cite{Commission1998}.

\subsection{Motivation}
Die erfolgreiche Entwicklung von Automatisierungssystemen im industriellen als auch privaten Sektor stellt die Grundlage für den wirtschaftlichen Erfolg im Konkurrenzkampf der großen Industrienationen dar. Sowohl die Qualität des Endprodukts als auch die Effektivität des Prozesses ist entscheidend für den Mehrwert eines Systems. Somit sollte bereits in der technischen Ausbildung zukünftiger Ingenieure das Vorgehen zum anforderungsorientierten Entwickeln von Systemen und deren Software im Mittelpunkt stehen \cite[V]{Andelfinger2017}.\\
Trotz der Erkenntnis, dass Top-Down Design durch Objekt-orientiertes Design zu ersetzen gilt, haben vor allem Unternehmen, deren Wurzeln im Hardwareentwurf liegen, Probleme beim Softwareengineering in der Automatisierungstechnik. Besonders die Projektanfangsphase, welche im englischen als Requirements Engineering bezeichnet wird, ist entscheidend für den Erfolg der Umsetzung eines solchen Systems \cite[1]{Laplante2014}.

\subsection{Problemstellung}
Im Rahmen eines Praktikums an der Hochschule für Technik und Wirtschaft Berlin (im Folgenden abgekürzt mit HTW) ist über einen Zeitraum von einem Jahr ein Positioniersystem in einem Laborraum der Hochschule konstruiert worden. Dieses besteht aus zwei Achsen und soll über einen Motioncontroller gesteuert werden.\\
Die Arbeit befasst sich mit der Konzeption des Automatisierungssystemes, mit dessen Projektierung und der Inbetriebnahme. Dabei soll die Anlage anforderungsorientiert konzipiert werden.

\subsection{Zielsetzung und Erkenntnisse}
Ziel der Arbeit ist es, ein Positioniersystem zu entwickeln, welches von jedem Laborplatz aus genutzt werden kann, um verschiedenste anlagenspezifische Testszenarien zu erproben, wie beispielsweise das Fahren von Trajektorien. Dabei sollen aus dem Prozessablauf Daten generiert werden können. Das bedeutet konkret, dass über einen oder mehrere Kommunikationskanäle von der Anlage aus Daten zur Verfügung gestellt werden, um diese anschließend extern weiterzuverwenden.\\
\textbf{Ziele der Arbeit:}
\begin{itemize}
	\item Analyse der Anforderungen an das Positioniersystem
	\item Aufstellen von Testkriterien für die ermittelten Anforderungen
	\item Entwickeln eines Konzeptes nach Handlungsempfehlungen bezüglich des Requirements Engineerings
	\item Darstellen der Anlagenprojektierung unter Nutzung von Modellen (UML)
	\item Implementieren der Modelle als Automatisierungssoftware mit dem \glqq MachineExpert LogicBuilder\grqq{} unter Beachtung der Norm DIN/EN 61131
	\item Inbetriebnahme des Systems unter Prüfung der festgelegten Testkriterien
	\item Bereitstellen von Prozess- und Maschinendaten für die Weiterverarbeitung
	\item Bereitstellen von Schnittstellen für die Bedienung und Programmierung von jedem Laborplatz
\end{itemize}
Es wird erwartet, dass die Ergebnisse der Arbeit zeigen, dass mithilfe des Requirements Engineerings und der Anlagenprojektierung erfolgreich ein Positioniersystem entwickelt werden kann, dass die Anforderungen an Selbiges erfüllt.\\
Dabei werden die Zugriffsmöglichkeit auf die Anlage von jedem Laborplatz und das Generieren sowie Bereitstellen von anlagenspezifischen Daten als die grundlegenden Anforderungen ermittelt.\\
Die Inbetriebnahme am Ende der Arbeitsphase beinhaltet das erfolgreiche Verifizieren der Testkriterien nach eventuellen Verbesserungen der implementierten Modelle aus der Projektierungsphase.

\subsection{Forschungsstand und theoretische Grundlage}
Die Thematik Requirements Engineerings wurde in der Literatur bereits behandelt und es existieren einige Quellen, die Handlungsempfehlungen zur Anwendung der Thematik beinhalten. Dennoch wird das Requirements Engineering noch zu selten angewendet, trotz der bekannten monetären und zeitlichen Vorteile \cite[xvii]{Laplante2014}. Die Methodik gilt als erfolgreiches Mittel, um mögliche Nachbearbeitungen von Software zu vermindern und die Kosteneffizienz zu verbessern \cite[1]{
Laplante2014}.\\
Grundlegend kann die Vorgehensweise in zwei Schritte unterteilt werden, die \glqq Artificial Development Sequence\grqq{} (künstliche Entwicklungssequenz) und die \glqq Engineering Activity Sequence\grqq{} (Aktivitätsentwicklungssequenz). Erste beschäftigt sich mit dem Aufstellen der Nutzer- und Systemanforderungen sowie mit den Systemdesign-Spezifikationen. Letztere beinhaltet das Testen der Anforderungen nach der Systementwicklung \cite[6]{
Laplante2014}.
Auch die Anlagenprojektierung ist in der Literatur bereits thematisiert worden und es existieren Empfehlungen für die Umsetzung dieser. Die Projektierung ist die Gesamtheit aller Entwurfs-, Planungs- und Koordinierungsmaßnahmen, mit denen die Umsetzung eines Automatisierungsprojektes vorbereitet wird \cite[8]{Bindel2017}.\\

\subsection{Aufbau der Arbeit}
Die Arbeit ist in drei Teile unterteilt.

\begin{itemize}
    \item \textbf{Konzeption}
    \item \textbf{Projektierung}
    \item \textbf{Inbetriebnahme}
\end{itemize}

Die Konzeption beinhaltet die Aufstellung der Systemintentionen aus den Anforderungen von den für das System relevanten Personen (Stakeholder). Die Dokumentation und Kategorisierung der Anforderungen findet in der Anforderungsanalyse statt.\\
In der Projektierung wird die Modellierung des Systemverhaltens vorgenommen. Dabei steht vor allem die Entwicklung der Systemsoftware im Mittelpunkt.\\
Die Inbetriebnahme ist der letzte Schritt in der Realisierung des Systems. Es gilt die Modelle aus der Projektierungsphase zu implementieren und die Erfüllung der Anforderungen unter Nutzung von Testkriterien zu verifizieren.\\

\end{document}

% \subsection{Zielstellung}
% Ziel dieser Arbeit ist die Entwicklung eines Positioniersystems, welches im Laborbetrieb von Studierenden genutzt werden soll. Die Umsetzung des Systems wird dabei in drei wesentliche Schritte unterteilt, den \textbf{Konzeptionsteil}, den \textbf{Projektierungsteil} und die \textbf{Inbetriebnahme}. Für die erfolgreiche und normkonforme Dokumentation und Nutzung der drei Kapitel bis hin zur Inbetriebnahme des positioniersystems als Laboranlage an der \ac{htw} Berlin, muss zunächst ein Grundgerüst gesetzt werden. Dieses besteht vor allem aus den beiden essenziellen Themen \textbf{Requierements Engineering} (zu deutsch Anforderungsentwicklung oder auch Anforderungsmanagement) und Anlagenprojektierung mit dem Fokus auf Automatisierungsanlagen.\\
% Die Konzeptionsphase ist besonders geprägt durch das Requierements Engineering. Insbesondere der Abschnitt der Anforderungsanalyse unter Einbeziehung der Stakeholder und der Testspezifikation(en) zu den Anforderungen, begründen sich auf diesem. Die nachfolgende Systemanalyse im Projektierungsteil der Arbeit basiert auf den Anforderungen, die nach den Handlungsempfehlungen des Requierements Engineerings dokumentiert wurden.\\
% Vor allem der zweite Teil dieser Arbeit, macht sich der Methodiken zur Projektierung von Automatisirungsanlagen zu Nutzen. Bereits die Systemanalysephase kann der Projektierung zugeornet werden. In dieser Arbeit liegt der Schwerpunkt im Kapitel der Projektierung jedoch insbesondere auf der Entwicklung der Automatisierungssoftware, bei der die Modellierung unter Zuhilfenahme von \acs{uml} Diagrammen zum Einsatz kommt. Für die Anlagenprogrammierung bedarf es der Einhaltung der gültigen Norm DIN/EN 61131, die auf der IEC Norm 61131 basiert und sich mit speicherprogrammierbaren Steuerungen befasst. Insbesondere Teil 3 der Norm beschäftigt sich mit den Programmiersprachen, die auch in der auf Codesys basierenden Entwicklungumgebung, welche in dieser Arbeit genutzt wird, zur verwendung kommen. \\
% Der letzte Teil der Arbeit, die Inbetriebnahme, stellt die Verifizierung der Anforderungen aus dem Konzeptionsteil dar. Es handelt sich also um das logische Endstück für das Vorgehen nach den Handlungsempfehlungen des Requierements Engineering. Endstück meint jedoch nicht, dass ein System als fertig umgesetzt gilt, wenn es in der Phase der Inbetriebnahme angelangt ist. Bei der Anforderungsanalyse handelt es sich um ein iteratives Verfahren, bei dem aus den Ergebnissen der vorherigen Iteration mögliche neue Anforderungen entstehen könnten.\\ % Quellen einbauen
% \bigskip \newline
% Im Zentrum der Entwicklung des Positioniersystems stehen vor allem zwei Themen. Zum einen soll die Laboranlage als Lehrmittel eingesetzt werden können, um Softwareentwicklung von Industriellen Systemen zu erlernen. Dies schließt die Nutzung von industriellen Kommunikationsstandards mit ein. Besonders die Implementation von Positionieranwendungen, so wie sie in Realität genutzt werden könnten, soll im Labor geübt werden können. \\
% Auch die zweite Thematik, die Digitalisierung von Prozessen, spielt im Zusammenhang mit dem mehrachsigen Positioniersystem eine wichtige Rolle. Dazu soll eine Schnittstelle implementiert werden, über welche das System Daten bereitstellen kann, die von einem anderen System weiterverwendet werden können. Dies schließt die bidirektionale Nutzung der Kommunikation ein, so dass die Laboranlage von externen Akteuren gesteuert werden kann.\\
% Ergebnis der Arbeit soll ein Echtzeitsystem sein, welches in Form einer Positioniereinheit von jedem Laborplatz aus programmiert und genutzt werden kann. Durch die implementierte Kommunikationsschnittstelle ist es möglich das System als Eingebettetes System in ein größeres Arbeitsspektrum einzugliedern. Es wird möglich sein über Verwaltungsschalen einen digitalen Zwilling zu generieren, der die dezentrale Steuerung und Verwaltung von Systemresourcen ermöglicht. 