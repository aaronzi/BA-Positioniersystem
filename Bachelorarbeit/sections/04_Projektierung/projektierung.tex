\documentclass[../../Bachelorarbeit.tex]{subfiles}
\begin{document}

\section{Projektierung}
Nachdem im letzten Kapitel die Anforderungsphase des mehrachsigen Positioniersystems behandelt wurde, schließt sich nun die Design- \bzw Modellierungsphase in diesem Kapitel an.\\
Der Entwicklungsprozess unter den Gesichtspunkten der Projektierung umfasst folgende fünf Kernabschnitte, die es zu untersuchen gilt:

\begin{itemize}
    \item \textbf{Kontextanalyse:} Finden der Systemgrenzen und Ermittlung von Nachbarsystemen.
    \item \textbf{Anwendungsfallspezifikation:} Identifizierung der Systemprozesse und anschließende Präzisierung.
    \item \textbf{Verhaltensspezifikation:} Modellierung des Systemverhaltens.
    \item \textbf{Partitionierung:} Untergliederung des Systems in logische Sinnesabschnitte zur Verringerung der Komplexität.
    \item \textbf{Testspezifikationen:} Festlegung von Prüfkriterien zur Bestätigung der Anforderungsumsetzung.
\end{itemize}

Die drei letzten Abschnitte des Kapitels zur Projektierung dienen als direkte Vorbereitung für die Implementationsphase das Positioniersystem. Es wird zum einen kurz die Entwicklung des Stromlaufplanes dargestellt, welcher die Grundlage bildet für die elektrotechnische Inbetriebnahme der Laboranlage. Mit beschrieben in diesem Unterabschnitt wird auch die Netzwerkstruktur zwischen sowohl den Steuerungskomponenten unter sich als auch (übergeordneten) Nachbarsystemen. \\
Nachfolgend soll in tabellarischer Form das Datenmodell aufgestellt werden, welches zusammen mit den UML Diagrammen aus dem Entwicklungsprozess die Basis für die Softwareimplementation darstellt.\\
An letzter Stelle in Projektierungsteil der Arbeit soll das Bedienkonzept entwickelt werden. Dabei wird sowohl auf die Bedienung über Eingabetaster an der Anlage selbst eingegangen, als auch eine dezentralisierte Bedienung über Netzwerkfähige Endgeräte, wie unter anderem die Laborcomputer, die sich im selben Raum wie das Positioniersystem befinden.

\end{document}