\documentclass[../../Bachelorarbeit.tex]{subfiles}
\begin{document}

\section{Zusammenfassung und Fazit}
% \color{red}
% Fazit muss unbeding noch hinzugefügt werden !!!
Über den Verlauf der Arbeit konnte ein mehrachsiges Positioniersystem wie auch dessen Systemsoftware entwickelt werden. Dabei wurden folgende Zielstellungen abgearbeitet:

\bigskip
\begin{compactitem}
	\item Erstellung einer Anforderungsanalyse des mehrachsigen Positioniersystems
	\item Aufstellung von Testkriterien zu den Anforderungen an das System
	\item Erstellung eines Anlagenkonzeptes unter Handlungsempfehlungen durch das Requirements-Engineering
	\item Modellierung des Systemverhaltens mithilfe der UML unter den Gesichtspunkten der Anlagenprojektierung
	\item Implementierung der Modelle unter Beachtung der Norm DIN/EN 61131 als Steuerungsprogramm des Systems
	\item Erfolgreiche Inbetriebnahme des Positioniersystemsystems unter Prüfung der festgelegten Testkriterien
	\item Bereitstellen von Prozess- und Maschinendaten für die Weiterverarbeitung und dezentrale Steuerung
	\item Bereitstellen einer Schnittstelle für die Bedienung und Programmierung des Systems von allen Laborplätzen
\end{compactitem}
\bigskip

Dazu werden beginnend mit dem Konzeptionsteil der Arbeit alle Anforderungen der verschiedenen Stakeholder das System betreffend ermittelt. Diese können anschließend unterteilt werden in funktionale und nicht-funktionale Anforderungen, die in der anschließenden Projektierungsphase als geforderte Funktionalität modelliert werden müssen. In der Konzeption werden weiterhin die Stakeholder ermittelt um verschiedene Interessengebiete festzustellen und Anforderungen gewichten zu können. Mithilfe einer Konfiguratorgrafik können sämtliche Systemkomponenten zusammengetragen und skizzenhaft in Zusammenhang gebracht werden, um einen Grobüberblick herzustellen. Im Konzeptionsteil wird weiterhin ein Bedienkonzept entworfen, welches Grundlage für die spätere Umsetzung sein soll.\\
Aus dem entwickelten Konzept kann im zweiten Teil der Arbeit, der Anlagenprojektierung, das weitere Vorgehen abgeleitet werden. Beginnend mit der Kontextanalyse findet die Modellierung des Systems und seines Verhaltens statt. In der Kontextanalyse werden zunächst die Systemgrenzen sowie Nachbarsysteme ermittelt. Anschließend findet eine Identifizierung der Systemprozesse und deren Präzisierung in der Anwendungsfallspezifikation statt. Basierend auf der Anwendungsfallspezifikation kann im nächsten Analyseschritt das Systemverhalten modelliert werden. Zur Komplexitätsverringerung wird daraufhin eine Partitionierung vorgenommen, um die Entwicklung zu vereinfachen und übersichtlicher zu gestalten. Im letzten Schritt der detaillierten Systemanalyse werden sämtliche Testspezifikationen der vorhergegangenen Analyseschritte zusammengetragen und strukturiert dokumentiert. Nach der Implementation müssen unter Verifizierung der aufgestellten Testkriterien die Anlagenfunktionen geprüft werden. Nachfolgend wird in der Projektierungsphase der Stromlaufplan des Systems entwickelt, auf dessen Grundlage die Hardwareimplementation nachfolgend stattfindet. Das Datenmodell stellt die Grundlage für die Softwareimplementation dar und bildet eine Brücke zwischen den realen Komponenten, wie Aktuatoren und Sensoren und Variablen im Steuerungsprogramm. Abschließend in der Systemmodellierung spielt die funktionale Sicherheit eine wichtige Rolle, um den Schutz von Mensch und Anlage sicherzustellen. Es wird eine Zusammenstellung von Sicherheitsmaßnahmen vorgenommen.\\
Im Kapitel zu Inbetriebnahme werden beginnend mit der Implementationsphase die Modellierungen aus der Projektierung des Positioniersystems implementiert. Begonnen mit der Hardware findet die Dokumentation der Montage- und Verdrahtungsaufgaben statt. Daran anschließend kann die Systemsoftware implementiert werden, die im Kontext der Arbeit eine besonders wichtige Rolle spielt. Nach Abschließen der Implementationsphase müssen mithilfe von Testfällen die in der Anforderungsanalyse ermittelten Anforderungen \bzw deren Testkriterien überprüft werden. Treten Fehler auf oder können Anforderungen nicht erfüllt werden, müssen diese je nach Gewichtung korrigiert \bzw verbessert werden.\\
\smallskip \newline
Resultat der Arbeit ist die erfolgreiche Entwicklung eines mehrachsigen Positioniersystems, welches sowohl als Lehrmittel als auch als eingebettetes System im Automatisierungs- \bzw Prozesssteuerungssysteme-Labor eingesetzt werden kann. Es besteht die Möglichkeit Anlagenspezifische Funktionen zu nutzen und die Programmierung dieser als Softwareentwicklung zu üben und zu erlernen. Durch die Implementierung eines OPC Servers auf der Systemsteuerung können sowohl Daten aus dem Systemprozess \bzw der Systemsoftware bereitgestellt werden, als auch Daten empfangen werden. Dies ermöglicht auf der einen Seite einen Mehrwert aus den generierten Daten zu erzeugen, in dem diese visualisiert und/oder analysiert werden. Relevanz hat das für sowohl das Nachbarsystem Augmented Reality Server, welches einen schnellen Überblick über das Positioniersystem verschaffen soll, als auch für die Integrierung der Daten in eine Verwaltungsschale, welche das System digital repräsentiert. Auf der anderen Seite kann durch das zurückschreiben von \zB Positionsdaten die Positioniereinheit dezentral gesteuert werden. Es ergibt sich die Nutzung des Systems als Smart-Application. \\
Vor allem durch die Messung und Bereitstellung von Leistungsdaten des Systems durch die Wago Steuerung können Optimierungen am Fahrverhalten vorgenommen werden. Dies ermöglicht eine quantitative Bewertung der Qualität der entwickelten Systemsoftware.\\
Dennoch wurden in der Testphase Fehler \bzw Mängel ermittelt, die auszubessern sind. Da die Kernanforderungen jedoch erfüllt sind, kann die Aufgabenstellung zur Entwicklung des Positioniersystems als erfüllt bewertet werden. 

\end{document}