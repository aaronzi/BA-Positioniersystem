\documentclass[../../Bachelorarbeit.tex]{subfiles}
\begin{document}

\section{Konzeption}
% Text und Grafik Systemprozess nach Literatur Prozessautomatisierung und uml arbeit seite 49
Dieses Kapitel unterteilt sich in vier Abschnitte. Die Konzeptionierung der Laboranlage erfolgt nach dem Requirements-Engineering, welches als Grundlage im vorhergegangenen Kapitel bereits kurz behandelt wurde.\\
Im ersten Unterkapitel wird die Anforderungsanalyse durchgeführt. Schwerpunkt der Arbeit liegt dabei auf dem Software- und Systementwicklungsprozess.\\
Der Entwicklungsprozess unter dem Gesichtspunkt der Konzeption des Systems umfasst dabei zwei Kernabschnitte, welche sich in die Anforderungsphase des Entwicklungsprozesses eingliedern. Bei den Kernabschnitten handelt es sich um folgende Analyseschwerpunkte.\\
\begin{itemize}
    \item \textbf{Anforderungsanalyse:} Es wird unterschieden zwischen funktionalen und nicht-funktionalen Anforderungen an das System.
    \item \textbf{Identifikation der Stakeholder:} Ermittlung aller an der Systementwicklung und Systemnutzung beteiligten Personen zur Feststellung von Randbedingungen an die Anforderungen.
\end{itemize}

Der dritte Abschnitt dient zum Grobentwurf des Hardwareaufbaus des Positioniersystems. Dazu wird eine Konzeptgrafik entwickelt, die als Konfigurator zu verstehen ist. Weiterhin werden die geplanten Betriebsmodi beschrieben, die in der Projektierung dieser Arbeit zu modellieren sind.\\
Der letzte Abschnitt der Konzeptionsphase beinhaltet die Erstellung eines Bedienkonzeptes für die Laboranlage. Dabei wird sowohl auf die Bedienung über Eingabetaster an der Anlage selbst eingegangen, als auch eine dezentralisierte Bedienung über netzwerkfähige Endgeräte, wie unter anderem die Laborcomputer, die sich im selben Raum wie das Positioniersystem befinden.

\end{document}