\documentclass[../../Bachelorarbeit.tex]{subfiles}
\begin{document}

\section{Konzeption}
Dieses Kapitel unterteilt sich in acht Abschnitte. Die Konzeptionierung der Laboranlage erfolgt nach dem Requirements-Engineering, welches als Grundlage im vorhergegangenen Kapitel bereits behandelt wurde.\\
Im ersten Unterkapitel wird die umzusetzende Anlage vorgestellt. Daran anschließend steht im Mittelpunkt der Entwicklungsprozess zum Entwurf des mehrachsigen Positioniersystems. Schwerpunkt der Arbeit liegt dabei auf dem Software- und Systementwicklungsprozess, während der Hardware- und Hardwareentwicklungsprozess in verkürzter Form Erwähnung findet (jeweils am Ende des Unterkapitels).\\
Der Entwicklungsprozess unter dem Gesichtspunkt der Konzeption des Systems umfasst dabei sieben Kernabschnitte, welche sich in die Anforderungsphase des Entwicklungsprozesses eingliedern. Bei den Kernabschnitten handelt es sich um folgende Analyseschwerpunkte.\\
\begin{itemize}
    \item \textbf{Anforderungsanalyse:} Es wird unterschieden zwischen funktionalen und nicht-funktionalen Anforderungen an das System.
    \item \textbf{Identifikation der Stakeholder:} Ermittlung aller an der Systementwicklung und Systemnutzung beteiligten Personen zur Feststellung von Randbedingungen an die Anforderungen.
    \item \textbf{Kontextanalyse:} Finden der Systemgrenzen und Ermittlung von Nachbarsystemen.
    \item \textbf{Anwendungsfallspezifikation:} Identifizierung der Systemprozesse und anschließende Präzisierung.
    \item \textbf{Verhaltensspezifikation:} Modellierung des Systemverhaltens.
    \item \textbf{Partitionierung:} Untergliederung des Systems in logische Sinnesabschnitte zur Verringerung der Komplexität.
    \item \textbf{Testspezifikationen:} Festlegung von Prüfkriterien zur Bestätigung der Anforderungsumsetzung.
\end{itemize}

\end{document}