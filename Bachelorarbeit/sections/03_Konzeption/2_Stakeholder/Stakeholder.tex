\documentclass[../../../Bachelorarbeit.tex]{subfiles}
\begin{document}
\clearpage
\subsection{Identifikation der Stakeholder} \label{stakeholder}
Wie in Grafik 1 zu erkennen ist, gehört die Ermittlung der für das Projekt wichtigen Personen (nachfolgen als Stakeholder bezeichnet) zur Anforderungsanalyse. Als Stakeholder gelten Personen, die an der Systementwicklung beteiligt sind aber auch zukünftige Anwender \bzw Personen, die vom Einsatz des Systems betroffen sind. Die in den vorangegangenen Unterkapiteln aufgenommenen Anforderungen und Randbedingungen haben ihre Grundlage auf den von den Stakeholdern bereitgestellten Informationen. Dabei vertreten die Stakeholder verschiedene Interessen das zu entwickelnde System betreffend.\\ % Quelle
Wie auch schon bei den nicht-funktionalen Anforderungen festgehalten wurde, können unterschiedliche und sogar konträre Bedürfnisse und Ansprüche von den Stakeholdern aufgestellt werden. Um den Entwicklungsprozess durch wiedersprüchliche Anforderungen von Stakeholdern nicht behindern zu lassen, wird eine tabellarische Auflistung aller für das System relevanten Personen erstellt. Es findet eine Klassifiezierung der Stakeholder statt, aus der ersichtlich wird, welche Person \bzw welcher Personenkreis für eine bestimmte Thematik als Ansprechpartner gilt. Anforderungen aus aus einem bestimmten Themenfeld werden priorisiert, wenn diese von Personen des selbigen Themenfeldes gestellt wurden. \autoref{tab:my-table3} zeigt die Auflistung aller Stakeholder des mehrachsigen Positioniersystems. Besonders wichtige Spalten der Tabelle sind zum einen die Rolle des oder der Stakeholder(S), der \bzw die Vertreter und deren Wissengebiet.\\

\begin{longtable}[C]{| R{0.19\linewidth} | R{0.18\linewidth} | R{0.15\linewidth} | R{0.17\linewidth} | R{0.17\linewidth} |}
    \hline
    \textbf{Rolle der Stakeholder} & \textbf{Beschreibung} & \textbf{Konkrete Vertreter} & \textbf{Wissengebiet} & \textbf{Begründung} \\ \hline
    Lehrpersonal der Hochschule & Auftraggeber für den eigentlichen Einsatz des Systems & Herr Prof. Dr. Schäfer \newline\medskip {\tiny Tel.: 5019-3466 \newline E-Mail: Stephan.Schaefer @HTW-Berlin.de\par}  & Lehre und Forschung im Gebiet der Automatisierungstechnik & Auftraggeber und Verantwortlicher \\ \hline
    Mitarbeiter der Labore für Automatisierung & Geben zusätzliche Anforderungen für die Verwendung vor & Herr Dipl. Ing. Dirk Schöttke \newline\medskip {\tiny Tel.: 5019-3564 \newline E-Mail: Dirk.Schoettke @HTW-Berlin.de\par}  & Ingenieur mit Fachkenntnissen in der Automatisierungstechnik sowie Anlagenprojektierung & Sorgt für die Eingliederung des Systems in Übergeornete Projekte \\ \hline
    Studenten der Hochschule & Sind die eigentlichen benutzer des Systems & keine Vertreter & Arbeiten mit den Laboranlagen des Fachbereiches & Müssen das System im Lehrbetrieb der Hochschule benutzen \\ \hline
    Prozessentwickler & Person(en) die für die Entwicklung des Systems verantwortlich ist/sind & Herr Aaron Zielstorff \newline\medskip {\tiny tel.: +49177/2847470 \newline E-Mail: Aaron.Zielstorff @HTW-Berlin.de\par} & Entwickler des Positioniersystems & Ist verantwortlich für die Realisierung des Systems nach gegebenen Anforderungen \\ \hline
    \caption[Stakeholder]{Stakeholder des mehrachsigen Positioniersystems}
    \label{tab:my-table40}
\end{longtable}

Die Auflistung der Stakeholder ergibt, dass grundsätzlich zwei Interessengebiete und somit auch zwei Interessensgemeinschaften existieren, was die Anforderungen und Interessen an das Positioniersystem betrifft. Auf der einen Seite soll die Positioniereinheit im Lehrbetrieb im Labor eingesetzt werden, um vorlesungsbegleitend Studierenden die Möglichkeit zu bieten, praxisnah Industriesteuerungen für Bewegungsaufgaben (Motion Controlling) zu programmieren. Dazu zählt auch die Anforderung von jedem Laborplatz aus Automatisierungssoftware zu entwickeln, die nach Fertiggstellung auf die Steuerung des mehrachsigen Positioniersystems übertragen werden kann, um Positionieraufgaben an einem realen System durchzuführen. Ziel soll es sein dadurch nicht nur rein simulativ die Abläufe bei der Inbetriebnahme eines solchen Systems zu erproben, sondern zusätzlich auch reale physikalische Einflüsse mit zu berücksichtigen, die eventuelle Abweichungen zu simulierten Automatisierungsprogrammen aufweisen. Außerdem müssen bei der Nutzung realer Hardware auch die Sicherheit von Mensch uns Maschine mit berücksichtigt werden, da Gefahren durch den Betrieb des Systems auftreten können.\\
Auf der anderen Seite soll das Positioniersystem in Drittpjojekten mit eingebettet werden. Dazu ist es von Relevanz, dass wichtige Systemdaten über einen OPC UA Server (diese Rolle wird von der Steuerung übernommen) bereitgestellt werden. Über diese Schnittstelle können Daten aus dem Prozess abgegriffen werden, die in einem externen System \bzw in einem Peripheriesystem weiterverarbeitet oder genutzt werden können.\\
Zusammenfassend kann festgehalten werden, dass die Positioniereinheit Wertschöpfung als Lehrmittel für Studierende und als Quelle für relevante Sytemdaten zu diversen Zwecken generieren soll. Dazu ist es unabdinglich, dass sie über ethernetbasierte Schnittstellen (Netzwerkschnittstellen) mit externen Geräten kommunizieren kann. 

\end{document}