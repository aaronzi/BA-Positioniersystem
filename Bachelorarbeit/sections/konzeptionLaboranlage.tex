\documentclass[../Bachelorarbeit.tex]{subfiles}
\begin{document}

\section{Konzeption}
Dieses Kapitel unterteilt sich in sieben Abschnitte.

\subsection{Vorstellung der Laboranlage}
In diesem Unterkapitel wird zunächst die Laboranlage vorgestellt, die im Verlauf der Arbeit unter den Gesichtspunkten des Requierement Engineerings und der Anlagenprojektierung nach Din EN 61131 konzipiert, projektiert und in Betrieb genommen werden soll. Im ersten Abschnitt wird das bereits elektrisch fertiggestellte Positioniersystem dargestellt. Im Mittelpunkt steht hierbei die erleuterung des Aufbaus und die Beschreibung der Funktionalität der Anlage. Der zweite Abschnitt behandelt die Eingliederung des Systems in seine Arbeitsumgebung. Dabei soll ein erster Überblick zum Einsatz der Positioniereinheit gegeben werden. Eine wichtige Rolle für die Nutzung der Anlage spielen die verschiedenen Betriebsmodi, die am Ende dieses Abschnittes behandelt werden.

\subsubsection{Aufbau des Positioniersystems}
Wie bereits aus dem Thema der Bachelorthesis erkenntlich ist, handelt es sich bei der behandelten Laboranlage um ein mehrachsiges Positioniersystem. Dieses besitzt zum Zeitpunkt der ersten Inbetriebnahme zwei Achsen (siehe Bild).\\
Die horizontale Achse des Systems ist fest an der Wand montiert und hat eine Länge von rund 1600mm (Fahrtweg). Vertikal montiert auf dieser befindet sich die beweglich gelagerte zweite Achse der Positioniereinheit. Diese besitzt die Möglichkeit lineare Bewegungen zwischen den Endlagesensoren der Horizontalachse durchzuführen. Bei der Befestigung an der waagerechten Achse handelt es sich um ein doppeltes Schlittensystem auf Rollen. Die Bewegung der Achse erfolgt über ein Gummiriemen, der fest an der Vertikalachse befestigt ist, und über Umlenkrollen und einen Servomotor an der Horizontalachse bewegt werden kann. Auf der senkrechten Achse befindet sich ein weiterer Schlitten, der ebenso beweglich gelagert ist und sich auf einem Fahrtweg von rund 2000mm zwischen den beiden Endlagen bewegen kann. Auf diesem ist ein simples Greifsystem angebracht, welches horizontale 180 Grad Schwenkbewegungen durchführen kann, und in der Lage ist, grundlegende Greifoperationen durchzuführen.\\
Für die Zuleitungen zu den auf den bewegten Anlagenteilen montierten Aktoren und Sensoren wurden Energieketten verbaut, sodass Kabel Prozesssicher miteführt und eine dauerhafte Strom-, sowie Datenversorgung aller Systemkomponenten gewährleistet werden kann. An den beiden äußersten Profilen (sowohl auf der linken als auch auf der rechten Seite der Anlage) sind Ablagepositionen vorgesehen, von \bzw auf welche simple Transportgüter aufgenommen und abgelegt werden können.\\
Auf der rechten Seite direkt neben der Positioniereinheit sind der Schaltschrank sowie die Speicherprogrammierbare Steuerung (im Folgenden als SPS bezeichnet) an der Wand montiert. Die Kabel der aktoren und Sensoren des Systems münden an der Unterseite des Schrankes, sowie die Stromzuleitung und sämtliche Aus- und Eingangsverbindungen zu bzw. von der SPS und dem sich neben dieser befindenden Servoantrieb. Auf der Vorderseite an der Tür des Schaltschrankes sind Bedienelemente aufgeschraubt, die für die Grundlegende Steuerung der Anlage benutzt werden.\\
Zur gewährleistung der Sicherheit von Mensch und Anlage sind an der Vorderseite des Systems sowohl ein Lichtvorhang als auch Not-Halt Bedienelemente montiert. Stromfrei kann das Positionierstystem über den Hauptschalter an der rechten Seite des Schaltschrankes geschaltet werden.

\subsubsection{Betriebsumgebung}
Nachdem im vorhergehenden Abschnitt bereits die Grundlegenden Funktionen und der Aufbau der Positioniereinheit dargestellt wurden, beschäftigt sich dieses Unterkapitel mit der Darstellung der Eingliederung des Systems in dessen Arbeitsumgebung. Weiterhin sollen die Betriebsmodie \glqq Handbetrieb\grqq{}, \glqq Automatikbetrieb\grqq{}, \glqq Adminmodus\grqq{} und das Verhalten im Leerlauf sowie im Not-Halt implementiert werden.\\
Aufgebaut befindet sich das mehrachsige Positioniersystem im Laborraum G422 der HTW Berlin am Campus Wilhelminenhofstraße. Dort wurde die Anlage im Rahmen meines Praktikums errichtet. Nachfolgen ist es Ziel der Bachelorthesis, diese Anlage für den Lehrzweck in betrieb zu nehmen. Konkret soll die Positioniereinheit für zwei Anwendungen eingesetzt werden.\\
Erstere gliedert sich direkt in die Lehreinheiten des Laborbetriebs im späten Bachelor und Masterstudium im Themenfeld Automatisierungstechnik ein. Jeder studentische Laborplatz besitzt die Möglichkeit sich mit dem System zu verbinden, um es mit Automatisierungssaftware, die in den Lehreinheiten entwickelt wird zu bespielen und diese zu Testen. Es soll die Möglichkeit bestehen, Trajektorien zu fahren, bei denen virtuelle Hindernisse umgangen werden, und Objekte von einem Ausgangspunkt zu einem Zielpunkt transportiert werden können. Dabei soll die Nutzung der realen Laboranlage weitaus mehr Wissenszuwachs durch reale Einflüsse bieten, als ein rein simulativer Aufbau eines digital ähnlichen Positioniersystems.\\
Die zweite Anwendung der zweiachsigen Positioniereinheit, ist Teil eines Laborübergreifenden Projektes, welches nicht Teil dieser Arbeit ist. Aus dessen Zielen ergeben sich weitere Anforderungen an die Laboranlage. Es sollen Daten aus dem Prozessablauf bereitgestellt werden, aus denen Wertschöpfung generiert werden kann, indem diese extern weiterverarbeitet werden. Dazu müssen weitere Schnittstellen im System bereitgestellt werden, um generierte Daten mit Peripheriegeräten auzutauschen.

\subsubsection{Betriebsmodi}
\textbf{Automatikbetrieb:} Prinzipiell bekommt die Anlage eine Aufgabe übermittelt, die sie vollautomatisch zyklisch durchführen soll. Es kommt ein Funktionsablauf zustande. So könnte beispielsweise von einer Ablageposition A ein Objekt gegriffen werden, um Hindernisse herum möglichst ohne viel Vibration ein Weg gefahren werden, so dass besagtes Objekt an einer Ablageposition B wieder losgelassen wird. Danach fährt die Anlage wieder zu Position A (diesmal möglicherweise anders da Vibrationen keine Rolle spielen) und nimmt das nächste Objekt auf.\\
Konkret wird die Anlage im ersten Schritt unter Spannung gesetzt, in dem der Hauptschalter (400V Ebene) betätigt wird. Dieser befindet sich auf der rechten Seite des Schaltschrankes. Darauffolgend muss im zweiten Schritt die Steuerung (LMC Pro von Schneider Electric) eingeschalten werden, sowie alle Betriebsmittel auf der 24V Ebene mit Strom versorgt werden. Dies geschieht über den grünen Ein-Taster, welcher sich auf der Front des Schaltschrankes befindet. Der Eingeschaltete Zustand wird über eine Lampe auf der Schaltschrankfront signalisiert. Als Netzteil dient das LXM62P Powersupply von Schneider Electric, welches 3-phasig an der Drehstromsteckdose des Laborraumes angeschlossen ist. Dieses stellt den Strom für die 24V Ebene bereit und versorgt den LXM62D double Drive von Schneider Electric. Mit einem Wahlschalter kann nun der Automatikmodus des Systems angewählt werden. Die erfolgreiche Auswahl des Automatikmodus wird über eine Signalleuchte, welche mit „Auto“ betitelt ist, indiziert. 
Nach der Wahl des Automatikmodus fährt der Greifer, welcher sich auf der Z-Achse befindet, welche sich wiederum auf der X-Achse befindet, aus der sicheren Ausgangsposition (welche auch im ausgeschalteten Zustand vorliegt) zur Ablageposition A. Dazu werden zunächst die Bremsen der beiden Motoren gelöst, welche für jeweils die Bewegung einer Achse verantwortlich sind. Ist die Position vor der Ablagestelle A erreicht, wird im Nächsten Schritt ein Schwenkarm mit Greifer so zur Ablageposition A rotiert, dass ein sich darauf befindliches Objekt gegriffen werden kann. Es folgt besagter Greifprozess, um das auf Ablageposition A befindliche Objekt in Obhut des Positioniersystems zu bringen.\\
Die Anlage sieht sich nun einem Fahrtweg gegenüber, welcher von Hindernissen durchbrochen ist. Es ist ihr nicht möglich eine geradlinige Bewegung von Startposition A zur Zielposition, dem Ablageort B, zu fahren. Es ist weder möglich erst die komplette Bewegung in Z-Richtung auszuführen, und dann die Bewegung in X-Richtung, noch eine geradlinige Bewegung, so dass die Z- und X-Koordinate des Zieles gleichzeitig erreicht werden. Der Anlage bzw. der Software dieser sind die Positionen der Hindernisse bekannt.\\
Im nächsten Schritt muss sie einen Weg übermittelt bekommen, welcher ihr von der Automatisierungssoftware für das System zur Verfügung gestellt wird. Die Software berechnet aus der Position der Hindernisse einen Fahrtweg um diese herum. Dazu werden Splines generiert, so dass der Abstand zu den Hindernissen möglichst maximal ist und der Biegeradius der Splines möglichst groß, um Vibration und große Beschleunigungen in eine Richtung zu vermeiden. Das Objekt kann nach Übermittlung des Fahrtweges an das Positioniersystem nun von Ablageposition A zu Ablageposition B transportiert werden. 
Dort angekommen schwenkt der sich auf der Z-Achse befindliche Arm um, und das Objekt wird vom Greifer losgelassen, so dass es auf der Zielposition liegen bleibt. Die Anlage fährt nun den Weg wieder zurück zu Ablageposition A, um ein nächstes Objekt aufzunehmen und dieses wie schon beim ersten Objekt von Position A zu Position B zu transportieren.\\
Mögliche spätere Erweiterungen könnten sein, dass der Rückweg anders gewählt wird, da kein Objekt transportiert wird und somit auftretende Schwingungen bzw. Vibrationen keine wichtige Rolle spielen. Alternativ könnte auch auf dem Rückweg ein anderes Objekt von B nach A transportiert werden, welches andere Anforderungen aufweist.\\
Für die Relevanz des Automatikbetriebes ist es sinnvoll, wenn auch die Ablageposition(en) automatisch mit neuen Transportobjekten bestückt werden. Es würde sich eine spätere Erweiterung lohnen, die Förderbänder zu/von den Ablagepositionen der Anlage hinzufügt, so dass neue Objekte eingespeist werden in das System und bereits Transportierte herausgefahren werden.\\
Im letzten Schritt kann die Anlage wieder deaktiviert werden, was über die Abwahl des aktuellen Betriebsmodus geschieht. Dies ist in jedem Moment während des Automatikbetriebes möglich. Die letzte Transportaufgabe wird noch vollständig zu Ende durchgeführt. Danach wird ein sicherer Ausgangszustand angefahren und die Bremsen der Motoren werden wieder aktiviert, so dass die Z-Achse nicht herunterfallen kann. Nach erfolgreicher Abwahl des Betriebsmodus erlischt die Indikatorlampe für diesen Modus wieder. Nur wenn kein Modus ausgewählt ist, kann die 24V Ebene wieder spannungsfrei geschalten werden und die Steuerung somit wieder deaktiviert werden. Dieser geschieht über den roten Aus-Taster auf der Front des Schaltschrankes. Nach Betätigung des Tasters erlischt die Lampe, welche die Betriebsbereitschaft des Positioniersystems signalisiert.\\
Während der gesamten Betriebsdauer gelten mehrere Safety-Bedingungen. Es befinden sich mehrere Not-Aus Taster (einrastend) an der Anlage (am Schaltschrank, auf der Linken Seite an einem Profil). Die Betätigung führt zu einem sofortigem halt der Bewegung beider Achsen und dem Anzug der Bremsen in den Motoren. Gleiche Funktionalität tritt auf, wenn der Lichtvorhang von einem Objekt (menschlich oder nicht menschlich) durchbrochen wird. Nach Auftreten dieses Ereignis muss die Anlage wieder freigeschalten werden. Der aktuelle Betriebsmodus wird abgewählt, was dafür sorgt, dass die Anlage im Fall des Automatikmodus nun den letzten bearbeitungsschritt fertig zu Ende durchführt und dann in die Ausgangsposition fährt. Alternativ kann auch der Admin-Modus aktiviert werden, um die letzte Aufgabe zu unterbrechen/beenden und das Positioniersystem manuell zu bedienen.\\
Weiterhin muss während der gesamten Betriebsdauer eine Achsbewegung innerhalb der magnetischen Näherungssensoren gegeben sein. Wird eine Endlageposition auf einer der beiden Achsen überschritten, wird umgehend die Bremse des Motors der jeweiligen Achse aktiviert, so dass diese nicht über die Endposition hinausfahren kann und Beschädigungen verursacht. Da es sich bei der Bewegung der Achsen um eine geregelte Bewegung über Servos handelt kann davon ausgegangen werden, dass es sich bei der Aktivierung der Näherungsschalter um einen fehlerhaften Zustand handelt. Es wird analog zu der Not-Halt Funktionalität im vorhergehenden Absatz gehandelt.\\
Als letztes wird noch die rote und grüne Ampel/Signalsäule erläutert, welche rechts oben an einem Profil der Anlage befestigt ist. Diese dient zur Signalisierung des Betriebes der Anlage. Sobald sich eine Achse in Bewegung befindet, blinkt die Ampel. Dies dient als Warnung für umstehende Menschen, dass diese sich im Bereich der Anlage vorsichtig und achtsam verhalten sollten. Im Fehlerfall leuchtet die Ampel in ausschließlich roter Farbe, ist die Positioniereinheit betriebsbereit, leuchtet sie nur in grüner Farbe.\\
Weiterhin muss zu jedem Zeitpunkt, wo keine Bewegung einer Achse stattfindet die Bremse der jeweiligen Achse aktiviert sein.
\smallskip
\newline
\textbf{Handbetrieb:} Prinzipiell bekommt auch beim Handbetrieb die Anlage eine Aufgabe übermittelt, die durchgeführt werden soll. Anders als im Automatikbetrieb kann beim Handmodus nun jeder Zustand durch den Nutzer händisch erreicht werden. Das heißt konkret, dass Beispielsweise beim Transport eines Objektes von Ablageposition A zu Ablageposition B, erst durch Nutzerinteraktion die Anlage ein Objekt greift, dann weiterer Interaktion bedarf, um es zur Zielposition zu befördern und dort erneut eine Eingabe erforderlich ist, um das transportierte Objekt abzulegen. Dennoch ist die konkrete Aufgabe und der letztendlich durchgeführte Prozess genau der gleiche wie im Automatikbetrieb. Nützlich ist der Handbetrieb vor allem zum Testen von Systemfunktionalitäten, zur Optimierung von Beförderungsprozessen und für das Finden von möglichen Fehlern in der Regelungssoftware für die Achsbewegungen.\\
Konkret wird auch hier die Anlage im ersten Schritt wieder unter Spannung gesetzt durch Betätigung des Hauptschalters. Danach wird mit Hilfe des grünen Ein-Tasters die 24V Ebene aktiviert, wodurch auch alle Betriebsmittel, die mit dieser gekoppelt sind, eingeschalten werden. Abweichend zum Automatikmodus muss nun jedoch der Wahlschalter auf den Handbetrieb geschalten werden. Die erfolgreiche Auswahl wird durch das Aufleuchten der mit „Hand“ betitelten Lampe auf der Front des Schaltschrankes bestätigt.\\
Nach der Wahl des Handmodus verbleibt die Anlage zunächst im Ruhezustand an der sicheren Ausgangsposition. Um die Positioniereinheit in Bewegung zu setzen ist nun eine Nutzereingabe nötig. Anders als im nächsten Kapitel beschrieben, kommt nun der Vierwegeschalter als Eingabegerät für bestimmte Anlagenfunktionalitäten zum Einsatz. Dieser ist beim Handbetrieb keine direkte Steuerung, um das auf den beiden Achsen montierte Greifgerät in die jeweilige Richtung der Pfeile auf den Tastern zu bewegen, sondern eine Eingabemöglichkeit, um das Positioniersystem zwischen den einzelnen Ablaufzuständen wechseln zu lassen.\\
Die Taster mit den Pfeilen nach rechts und links veranlassen die Anlage zu einer Bewegung zum jeweils nächsten bzw. vorherigen Koordinatenpunkt im Funktionsablauf. Die Taster mit den Pfeilen nach oben und unten sind für die separate Steuerung des Greifers verantwortlich, welcher auf dem Schlitten der Z-Achse befestigt ist. Zum besseren Verständnis wird nun der Ablauf bzw. die Aufgabe, die auch schon im Kapitel zum Automatikbetrieb beschrieben wurde, analog nun für den Handbetrieb erläutert.\\
Durch Drücken der Pfeiltaste nach rechts fährt die Anlage mit den beiden Achsen den Greifer aus der Ausgangslage zur Ablageposition A. Dort kann der Nutzer nun durch Betätigung des oberen Tasters der Vierwegeeingabeeinheit den Schwenkarm mit dem Greifer über die Ablageposition drehen lassen. Durch erneutes Drücken würde sich der Arm um 180° drehen, um eine Greifaktion auf der anderen Seite durchführen zu können. Um ein Objekt aus der Ablageschale aufheben zu können ist nun die Eingabe des Tasters mit dem Pfeil nach unten nötig. Erneutes Betätigen würde den Greifer wieder öffnen, um das gegriffene Objekt wieder entlassen.\\
Genauso wie im Automatikbetrieb kann die Anlage keinen geradlinigen Weg zur Zielposition fahren, um das aufgenommene Objekt zu seiner Ablageposition zu bringen. Das Automatisierungsprogramm besitz wieder Informationen zu Bereichen, in denen sich Hindernisse befinden, die umfahren werden müssen. Anders jedoch als im Automatikmodus wird nun kein kompletter weg übermittelt, sondern nur einzelne Splines aus denen der Gesamtweg besteht. Somit wird jedes Hindernis einzeln umfahren, wenn es eine Wegkorrektur erfordert. Um eine Bewegung zu fahren muss der Nutzer mit der Pfeil nach rechts Taste die Anlage in Bewegung versetzen. Durch Drücken der linken Taste kann eine Bewegung rückgängig gemacht werden. Dass heißt die Anlage kann die bereits durchgeführten Fahrtwege umgekehrt wieder zurückfahren.\\
Je nach Menge und Position der Hindernisse wird nach einigen Eingaben des Nutzers der Zielort erreicht. Dort kann durch die obere bzw. untere Taste des Vierwegeschalters das transportierte Objekt abgelegt werden. Durch weiteres Interagieren mit der Vorwärts-Taste fährt die Positioniereinheit den Greifer wieder zur Aufnahmeposition A.\\
Beim Abwählen bzw. Deaktivieren des Handbetriebes ist es nicht vom Nutzer erforderlich, dass dieser die aktuelle Aufgabe händisch fertigstellt. Anstelle dessen fährt die Anlage direkt in den Ausgangszustand und aktiviert die Bremsen in den beiden Motoren. Zu jeder Zeit während des Betriebsablaufes im Handmodus kann auf den Automatikbetrieb umgeschaltet werden, was das Positioniersystem am aktuell vorliegenden Zustand ansetzen lässt, und von dort den Ablauf automatisch durchführt. Ab diesem Zeitpunkt gelten alle Regularien des Automatikbetriebes.\\
Die Safety-Funktionen des Handbetriebes sind annähernd identisch zu denen des Automatikmodus. Bei Betätigung eines Not-Aus Tasters oder durchbrechen des Lichtvorhangs stoppt die Positioniereinheit direkt alle Bewegungen und aktiviert die Bremsen in den Motoren. Gleiches geschieht auch bei Überschreitung der Endlagen der jeweiligen Achsen. Nach wieder Freigabe beendet die Anlage den aktuellen Teilschritt des Prozessablaufes und geht somit in den nächsten Zustand über. Von dort kann der Nutzer mit dem Handbetrieb fortfahren. Es ist nicht erforderlich den Betriebsmodus zur Freischaltung abzuwählen oder auf den Admin-Betrieb zu wechseln. (Die Möglichkeit besteht wie zu jedem anderen Zeitpunkt dennoch.)\\
Die Signalampel verhält sich wie im Automatikbetrieb, mit dem Zusatz, dass sie auch grün leuchtet, wenn auf die Eingabe des Nutzers gewartet wird, um den nächsten Zustand anzufahren.
\smallskip
\newline
\textbf{Adminmodus:} Auch der Admin-Modus kann nach Einschalten der Anlage über den Wahlschalter aktiviert werden. Signalisiert wird die Wahl des Modus über eine Lampe mit dem Titel „Admin“. Es handelt sich um denjenigen Modus der 3 Betriebsmodi, mit den wenigsten Restriktionen. Er soll zum einen als Möglichkeit dienen, aus Fehlerfällen wieder herauszufahren, kann aber auch für eine sehr individuelle händische Bedienung des Positioniersystems genutzt werden. Zu jeder Zeit während des Betriebsablaufes kann in den Admin-Betrieb umgeschaltet werden. Dabei ist es egal, ob sich Die Positioniereinheit bereits in einem anderen Betriebsmodus befindet oder einfach nur betriebsbereit ist.\\
Für das Umschalten von einem anderen Modus auf den Admin-Betrieb existieren verschiedene Regularien. Nach Auftreten eines Fehlers in egal welchem Betriebszustand kann immer direkt umgeschaltet und der Modus genutzt werden. Ist das System jedoch in der Abarbeitung eines Anwenderprozesses muss erst im Falle des Automatikmodus der aktuelle Zyklus fertig abgearbeitet werden, und im Falle des Handbetriebes, der nächste Zustand erreicht werden. Erst ab diesem Zeitpunkt ist der Admin-Betrieb durch den Nutzer möglich.\\
Im Nachfolgenden Absatz ist die Steuerung des Positioniersystems im Admin-Modus beschrieben. Zuerst sind jedoch zwei weitere Anmerkungen von großer Bedeutung. Das Umschalten aus dem Admin-Betriebsmodus in einen anderen Betriebsmodus ist nicht möglich. Die Anlage kann nur in den betriebsbereiten Zustand zurückgeschalten werden. Weiterhin sorgt das Auslösen eines Endlageschalters nicht für das Aufleuchten des Fehlerindikators der Signalampel. Aufgrund der Steuerungsbeschaffenheit im Admin-Modus wäre dies irreführend. Weiteres wird erklärt bei der Erläuterung der Bedienung im Admin-Betrieb. Die Ampel signalisiert aber weiterhin einen Fehler bzw. ein Halt-Event, wenn ein Not-Aus Taster betätigt oder der Lichtvorhang durchbrochen wurde.\\
Mit Hilfe des Vierwegeschalters können die Achsen des Positioniersystems bewegt werden. Die Bewegungen auf der X-Achse werden über die Taster links und rechts gesteuert, die Bewegungen auf der Z-Achse mit den Tastern oben und unten. Wird eine Achse bis zur Endlage bewegt stoppt sie und die Bremsen werden aktiviert. Bewegt sich die Anlage nicht, sind die Bremsen genauso aktiviert. Bewegt sich keine der Achsen wird die Betriebsbereitschaft durch ein grünes Licht auf der Ampel signalisiert. Bewegt sich mindestens eine Achse blinkt die Ampel zweifarbig. Gegengesetzte Bewegungen durch Tastendruck sind nicht möglich. Es wird immer die zuerst gedrückte Taste als Vorgabe genutzt.\\
Die Betätigung des eines Not-Aus Tasters oder das Eindringen in die Anlage durch den Lichtvorhang stoppt jegliche Bewegung, falls diese vorhanden war. Auch hier werden die Bremsen wieder aktiviert. Nach Freigabe kann das Positioniersystem direkt weiter genutzt werden im Admin-Betrieb.

\end{document}