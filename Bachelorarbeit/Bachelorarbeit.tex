\documentclass[12pt, a4paper, twoside]{article} %Dokumentenklasse Setzen

\usepackage[a4paper, left=2.5cm, right=2.5cm]{geometry} %Seitenrand setzen

\usepackage[T1]{fontenc}
\usepackage{lmodern}
\usepackage[utf8]{inputenc}
\usepackage[ngerman]{babel}
\usepackage{csquotes}
\usepackage{xpatch}
\usepackage{xspace} % setzten von Leerzeichen nach Abkürzungen
\usepackage{microtype} % für glättere Seitenränder

%Mathe Pakete
\usepackage{amsmath}
\usepackage{thmtools}
\usepackage{amsfonts}
\usepackage{amssymb}

%Listenumgebungen
\usepackage{listings}
\usepackage{paralist}
\usepackage{enumitem}

%Demo Text
\usepackage{blindtext}

% Farb-Pakete
\usepackage{xcolor}
\usepackage{colortbl}

% Für erweiterte Tabellen
\usepackage{longtable}

% Für Einheiten
\usepackage[exponent-product = \cdot]{siunitx}
\sisetup{locale=DE}

\makeatletter
\renewcommand\@dotsep{5}
\makeatother

% Pakete für Grafiken
\usepackage{graphicx}
\usepackage{wrapfig}
\usepackage{epstopdf}
\usepackage{subcaption}
%\captionsetup[subfigure]{list=true, font=normalsize, labelformat=brace, position=top} %setup für subfigure captions
\usepackage{pstricks}
\usepackage{pst-plot,pst-bar} %Balkendiagramme

% Paket für Literaturverzeichnis
\usepackage[
    style=alphabetic,
    sorting=nty,
    sortcites=true,
    autopunct=true,
    autolang=hyphen,
    hyperref=true,
    abbreviate=false,
    backref=true,
    backend=biber,
    block=space
    ]{biblatex}

\addbibresource{bib/bib.bib} %Einfügen der Literaturbibliothek
\defbibheading{bibempty}{}


\usepackage{url}
\usepackage{hyperref}
\hypersetup{hidelinks}
\urlstyle{same}

%Abkürzungen durch Kommandos setzen
\newcommand{\bspw}{bspw.\xspace}
\newcommand{\bzw}{bzw.\xspace}
\newcommand{\etc}{etc.\xspace}
\newcommand{\zB}{z.\,B.\xspace}
\newcommand{\EV}{e.\,V.\xspace}
\newcommand{\zT}{z.\,T.\xspace}
\newcommand{\iVm}{i.\,V.\,m.\xspace}
\newcommand{\idR}{i.\,d.\,R.\xspace}
\newcommand{\ihv}{i.\,H.\,v.\xspace}
\newcommand{\ua}{u.\,a.\xspace}
\newcommand{\dH}{d.\,h.\xspace}
\newcommand{\vgl}{vgl.\xspace}
\newcommand{\ca}{ca.\xspace}
\newcommand{\dV}{d.\,Verf.}
\newcommand{\RNr}{Rn.\xspace}
\newcommand{\oa}{o.\,{ä}.\xspace}
\newcommand{\vC}{v.\,Chr.\xspace}
\newcommand{\nC}{n.\,Chr.\xspace}
\newcommand{\vA}{v.\,a.\xspace}

\usepackage{tocbibind} %damit Verzeichnisse im Inhaltsverzeichnis aufgeführt werden
%\usepackage[notoc]{tocbibind} %damit Inhaltsverzeichnis nicht im Inhaltsverzeichnis vorkommt

\pagenumbering{Roman} %Römische Seitennummerierung für Verzeichnisse

\begin{document}

%TITELSEITE
\begin{titlepage}
	\begin{center}
	\vspace*{0.2cm}
	
	\huge
	\textbf{Konzeption, Projektierung und Inbetriebnahme eines mehrachsigen Positioniersystems}
	
	\vspace*{1.8cm}
	\normalsize
	im Studiengang\\
	\Large
	Elektrotechnik
	
	\vspace*{1.2cm}
	\normalsize
	am Fachbereich\\
	\Large
	Ingenieurwissenschaften - Energie und Information
	
	\vspace*{1.2cm}
	\normalsize
	an der\\
	\Large
	Hochschule für Technik und Wirtschaft Berlin
	
	\vspace*{2.0cm}
	\normalsize
	vorgelegt von\\
	\Large
	Aaron Zielstorff
	
	\vspace*{0.9cm}
	\normalsize
	Berlin, 04.06.2021
	
	\vspace*{2.0cm}
	\normalsize
	Betreuer:\\
	Herr Prof. Dr. Stephan Schäfer\\
	Herr Dipl.-Ing. Dirk Schöttke
	
	\end{center}
\end{titlepage}


%INHALTSVERZEICHNIS
\setcounter{tocdepth}{2} %Inhaltsverzeichnis zeigt 2 Gliederungsebenen
\tableofcontents
\thispagestyle{empty}
\clearpage

%ABBILDUNGSVERZEICHNIS
%\listoffigures

%TABELLENVERZEICHNIS
%\listoftables
%\clearpage

\pagenumbering{arabic} %Arabische Seitennummerierung nach den ab Textbeginn


\newpage
\section*{Literatur}
\addcontentsline{toc}{section}{Literatur}

\nocite{Winkelhake2021}
\nocite{Geisberger2012}
\nocite{Bauernhansl2014}
\nocite{Pistorius2020}

\printbibliography[
	heading=subbibintoc,
	type=book,
	title={Bücher}
]
	
%\addcontentsline{toc}{section}{Literatur}

\nocite{Wissenschaft2013}

\printbibliography[
	heading=subbibintoc,
	type=article,
	title={Artikel}
]

\end{document}